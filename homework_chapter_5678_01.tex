\documentclass[12pt,fleqn]{article}

\usepackage{ctex}
\usepackage{amsmath}
\usepackage{geometry}
\usepackage{nccmath}
\usepackage{amssymb}




\geometry{a4paper,left=1in, right=1in, top=1in, bottom=1in}


\title{\heiti 矩阵论第五、六、七、八章作业}
\author{XXX SZ2305XXX}
\date{2023年11月07日}

\renewcommand\arraystretch{0.8}
\renewcommand{\baselinestretch}{1.5}
\newcommand{\qedsymbol}{\hfill\square}


\begin{document}
	
	\maketitle
	
	\section*{\kaishu 第五章-习题11.}
		\subsection*{\kaishu {证:}}
		由题意知,$\boldsymbol{A}$,$\boldsymbol{B}$为$n$阶Hermite矩阵,且$\boldsymbol{A}$正定,由书本\textbf{定理5.2.1}可知存在n阶可逆Hermite矩阵$\boldsymbol{S}$使得$\boldsymbol{A}={\boldsymbol{S}}^2$.则
		\begin{ceqn}
			\begin{align*}
			\boldsymbol{A}\boldsymbol{B} &= \boldsymbol{S}^2\boldsymbol{B} \\
			\boldsymbol{S}^{-1}\boldsymbol{A}\boldsymbol{B}\boldsymbol{S} &= 	\boldsymbol{S}\boldsymbol{B}\boldsymbol{S}  \quad  \text{也是Hermite矩阵}\Bigl(\text{\textbf{5.1.1}Hermite矩阵的性质(5)}\Bigr).
			\end{align*}‎
		\end{ceqn}
		\indent 又由\textbf{定理5.1.3}可知,若$\boldsymbol{SBS}$是Hermint矩阵,则存在酉矩阵$\boldsymbol{U}$使得
		$$
		\boldsymbol{U}^H(\boldsymbol{SBS})\boldsymbol{U} = \boldsymbol{\Lambda} = diag(\lambda_1,\lambda_2,\cdots,\lambda_n)
		$$
		$\text{其中}\lambda_1,\lambda_2,\cdots,\lambda_n\text{均为实数}.$即
		$$
			\boldsymbol{U}^H\boldsymbol{SBSU} = \boldsymbol{\Lambda}
		$$
		代$\boldsymbol{S}\boldsymbol{B}\boldsymbol{S} = \boldsymbol{S}^{-1}\boldsymbol{A}\boldsymbol{B}\boldsymbol{S}$入得
		$$
			\boldsymbol{U}^H\boldsymbol{S}^{-1}\boldsymbol{ABSU} = \boldsymbol{\Lambda}
		$$
		其中$\boldsymbol{U}$为酉矩阵,则有$\boldsymbol{U}^{-1} = \boldsymbol{U}^H$,故上式化为
		$$
		\boldsymbol{U}^{-1}\boldsymbol{S}^{-1}\boldsymbol{ABSU} = \boldsymbol{\Lambda}
		$$
		$$
			\boldsymbol{(SU)}^{-1}\boldsymbol{(AB)(SU)} = \boldsymbol{\Lambda}
		$$
		记$\boldsymbol{P} = \boldsymbol{SU}$,则有$\boldsymbol{P}^{-1}\boldsymbol{(AB)}\boldsymbol{P} = \boldsymbol{\Lambda}$,即$\boldsymbol{AB}$相似于实对角矩阵$\boldsymbol{\Lambda}$.
		$\qedsymbol$
		\\ \\
	\section*{\kaishu 第五章-习题12.}
		\subsection*{\kaishu {证:}}
		由题意知,$\boldsymbol{A}$,$\boldsymbol{B}$为$n$阶Hermite矩阵,且$\boldsymbol{A}$正定,$\boldsymbol{B}$半正定,由书本\textbf{定理5.2.1}可知存在n阶可逆矩阵$\boldsymbol{Q}$使得$\boldsymbol{A}={\boldsymbol{Q}}^H{\boldsymbol{Q}}$.则
		$$
			\boldsymbol{AB} = \boldsymbol{Q}^H\boldsymbol{QB}
		$$
		因而有
		$$
			tr(\boldsymbol{AB}) = tr(\boldsymbol{Q}^H\boldsymbol{QB}) = tr(\boldsymbol{Q}^H\boldsymbol{BQ})
		$$
		其中$\boldsymbol{B}$半正定,由\textbf{定理5.2.2}(4)知$\boldsymbol{Q}^H\boldsymbol{BQ}$也为Hermite半正定矩阵,则由\textbf{推论5.2.2}(3)与\textbf{定理5.2.2}(4)得
		\begin{ceqn}
			\begin{equation}
				tr(\boldsymbol{Q}^H\boldsymbol{BQ}) \ge \lambda_i(i = 1,2,\cdots,n) \ge 0 \tag{*}
			\end{equation}
		\end{ceqn}
		\indent 又由题意知,
		\begin{ceqn}
			\begin{equation}
				tr(\boldsymbol{AB}) = tr(\boldsymbol{Q}^H\boldsymbol{BQ}) = 0 \tag{**}
			\end{equation}
		\end{ceqn}
		\indent 对比(*)式与(**)式,当且仅当$\lambda_i = 0 , \text{其中}i = 1,2,\cdots,n.$
		当且仅当$\boldsymbol{B} = \boldsymbol{0}$. $\qedsymbol$
		\clearpage
		
		
		
	\section*{\kaishu 第六章-习题18.}
		\subsection*{\kaishu {解:}}
		求幂级数$f(z) = \sum\limits_{i=1}^\infty c_k z^k$的收敛半径$R$:
		\begin{ceqn}
			\begin{align*}
				\lim_{k \to \infty} \vert \frac{c_k}{c_{k+1}} \vert = 
				\lim_{k \to \infty} \vert \frac{k^2}{(k+1)^2} \vert = 1
			\end{align*}
		\end{ceqn}
		则$R = \lim\limits_{k \to \infty} \vert \frac{c_k}{c_{k+1}} \vert = 1$. \\
		\indent 求矩阵$\boldsymbol{A} = \begin{pmatrix}
			1 & 4 \\ -1 & -3
		\end{pmatrix}$的谱半径$\rho(\boldsymbol{A})$:
		\begin{ceqn}
			\begin{gather*}
				| \lambda \boldsymbol{I} - \boldsymbol{A} | = 0 \\
				\begin{vmatrix}
					\lambda-1 & -4 \\ 1 & \lambda+3
				\end{vmatrix} = \lambda^2 + 2 \lambda + 1 = 0 \\
				\lambda_1 = \lambda_2 = -1 \\
				\rho(\boldsymbol{A}) = \max\limits_{1 \le i \le n} |\lambda_i| = 1
			\end{gather*}
		\end{ceqn}
		\indent 对比$R$与$\rho(\boldsymbol{A})$,有$R = \rho(\boldsymbol{A})$,故原矩阵幂级数收敛性无法判断.
		\clearpage
		
		
		
	\section*{\kaishu 第七章-习题1.}
		\subsection*{\kaishu {(1)解:}}
		$$
		\boldsymbol{A} = 
		\begin{pmatrix}
			2 & 0 & 0 \\ 0 & 1 & 1 \\ 0 & 0 & 1
		\end{pmatrix}
		$$
		注意到$\boldsymbol{A}$为Jordan标准形,有Jordan块
		$$
		\begin{array}{cc}
			\boldsymbol{J_1} = \begin{pmatrix} 2 \end{pmatrix} \quad & \quad
			\boldsymbol{J_2} = \begin{pmatrix} 1 & 1 \\ 0 & 1 \end{pmatrix}
		\end{array}
		$$
		则$\lambda_1 = 2 , \lambda_2 = 1$. \\[12pt]
		矩阵函数$f(\boldsymbol{A}) = \begin{pmatrix}
			f(\boldsymbol{J_1}) & \\ & f(\boldsymbol{J_2})
		\end{pmatrix}$,\\
		其中$f(\boldsymbol{J_i}) = 
		\begin{pmatrix}
			f(\lambda_i) & \dfrac{1}{1!}f'(\lambda_i) & \cdots & \dfrac{1}{(n_i-1)!}f^{(n_i-1)}(\lambda_i) \\
			 & f(\lambda_i) & \ddots & \vdots \\
			  & & \ddots & \dfrac{1}{1!}f'(\lambda_i) \\
			   & & & f(\lambda_i)
		\end{pmatrix}(i = 1,2; \  n_i \text{为}\boldsymbol{J_i}\text{的阶数})$ \\
		则有
		\begin{ceqn}
			\begin{gather*}
				f(\boldsymbol{J_1}) = \begin{pmatrix} f(\lambda_1) \end{pmatrix} \\
				f(\boldsymbol{J_2}) = \begin{pmatrix} f(\lambda_2) & f'(\lambda_2) \\
				 & f(\lambda_2)
				\end{pmatrix} 
			\end{gather*}
		\end{ceqn}
		$f(x) = e^x$时,$f'(x) = e^x$ ,$f(\lambda_1) = e^2$,$f(\lambda_2) = e$,$f'(\lambda_2) = e$
		$$f(\boldsymbol{A}) = 
		\begin{pmatrix}
			e^2 &  & \\
			 & e & e \\
			 & & e
		\end{pmatrix}
		$$ 
		$f(x) = e^{tx}$时,$f'(x) = te^{tx}$ ,$f(\lambda_1) = e^{2t}$,$f(\lambda_2) = e^t$,$f'(\lambda_2) = te^t$
		$$f(\boldsymbol{A}) = 
		\begin{pmatrix}
			e^{2t} &  & \\
			& e^t & te^t \\
			& & e^t
		\end{pmatrix}
		$$
		$f(x) = sin{tx}$时,$f'(x) = tcos{tx}$ ,$f(\lambda_1) = sin2t$,$f(\lambda_2) = sint$,$f'(\lambda_2) = tcost$
		$$f(\boldsymbol{A}) = 
		\begin{pmatrix}
			sin{2t} &  & \\
			& sint & tcost \\
			& & sint
		\end{pmatrix}
		$$
		\\
		
		
		
%		\subsection*{\kaishu {(2)解:}}
%		$$
%		\boldsymbol{A} = 
%		\begin{pmatrix}
%			2 & 2 & 1 \\ -2 & 6 & 1 \\ 0 & 0 & 4
%		\end{pmatrix}
%		$$
%		有
%		\begin{gather*}
%			\lambda \boldsymbol{I} - \boldsymbol{A} = 
%			\begin{pmatrix}
%				\lambda-2 & -2 & -1 \\
%				2 & \lambda-6 & -1 \\
%				0 & 0 & \lambda-4
%			\end{pmatrix}
%			\to
%			\begin{pmatrix}
%				-1 & \lambda-2 & -2 \\
%				-1 & 2 & \lambda-6 \\
%				\lambda-4 & 0 & 0
%			\end{pmatrix}	\\
%			\to
%			\begin{pmatrix}
%				-1 & \lambda-2 & -2 \\
%				0 & -\lambda+4 & \lambda-4 \\
%				0 & (\lambda-2)(\lambda-4) & -2(\lambda-4)
%			\end{pmatrix}
%			\to
%			\begin{pmatrix}
%				1 & 0 & 0 \\
%				0 & \lambda-4 & -(\lambda-4) \\
%				0 & (\lambda-2)(\lambda-4) & -2(\lambda-4)
%			\end{pmatrix}	\\
%			\to
%			\begin{pmatrix}
%				1 &  &  \\
%				& \lambda-4 &  \\
%				&  (\lambda-2)(\lambda-4) & (\lambda-4)^2
%			\end{pmatrix}
%			\to
%			\begin{pmatrix}
%				1 &  &  \\
%				& \lambda-4 &  \\
%				& & (\lambda-4)^2
%			\end{pmatrix}
%		\end{gather*}
%		\indent 则$\boldsymbol{A}$的初等因子为$\lambda-4$,$(\lambda-4)^2$,故$\boldsymbol{A}$的$Jordan$标准形为\par
%		\begin{ceqn}
%			\begin{align*}
%				\boldsymbol{J} =
%				\begin{pmatrix}
%					4 & 0 & 0 \\
%					0 & 4 & 1 \\
%					0 & 0 & 4
%				\end{pmatrix} ‎
%			\end{align*}‎
%		\end{ceqn}
%		其中Jordan块
%		$$
%		\begin{array}{cc}
%			\boldsymbol{J_1} = \begin{pmatrix} 4 \end{pmatrix} \quad & \quad
%			\boldsymbol{J_2} = \begin{pmatrix} 4 & 1 \\ 0 & 4 \end{pmatrix}
%		\end{array}
%		$$
%		则$\lambda_1 = \lambda_2 = 4$. \\[12pt]
%		\indent 求变换矩阵$\boldsymbol{P}$使得$\boldsymbol{P}^{-1}\boldsymbol{AP}=\boldsymbol{J}$,即求$\boldsymbol{P}=(p_1,p_2,p_3)$满足$\boldsymbol{AP}=\boldsymbol{PJ}$,带入$\boldsymbol{A}$、$\boldsymbol{P}$、$\boldsymbol{J}$得:
%		\begin{ceqn}
%			\begin{align*}
%				(\boldsymbol{A}p_1,\boldsymbol{A}p_2,\boldsymbol{A}p_3) = (p_1,p_2,p_3)
%				\begin{pmatrix}
%					4 & 0 & 0  \\
%					0 & 4 & 1  \\
%					0 & 0 & 4  
%				\end{pmatrix} ‎
%			\end{align*}‎
%		\end{ceqn}
%		\indent 比较上式两边得
%		\begin{ceqn}
%			\begin{align*}
%				\begin{cases}
%					\boldsymbol{A}p_1 = 4p_1,\\
%					\boldsymbol{A}p_2 = 4p_2,\\
%					\boldsymbol{A}p_3 = p_2 + 4p_3.
%				\end{cases} 
%			\end{align*}‎
%		\end{ceqn}
%		由此可见,$p_1$,$p_2$是$\boldsymbol{A}$的对应于特征值$4$的两个线性无关的特征向量.\par
%		\indent 从方程组
%		\begin{ceqn}
%			\begin{align*}
%				(\boldsymbol{A} - 4\boldsymbol{I}) x = 0 
%			\end{align*}‎
%		\end{ceqn}
%		可求得两个线性无关的特征向量$\xi = \begin{pmatrix}
%			1 \\ 0 \\ 2
%		\end{pmatrix}$,$\eta = \begin{pmatrix}
%			0 \\ 1 \\ -2
%		\end{pmatrix}$. \\
%		\indent 取$p_1 = \xi$,$p_2 = k_1\xi + k_2\eta$,其满足方程 $\boldsymbol{A}p_3 = p_2 + 4p_3$ 有解,记$p_3 = (x_1,x_2,x_3)^T$,即
%		\begin{ceqn}
%			\begin{align*}
%				\begin{pmatrix}
%					-2 & 2 & 1 \\
%					-2 & 2 & 1 \\
%					0 & 0 & 0
%				\end{pmatrix}
%				\begin{pmatrix}
%					x_1 \\ x_2 \\ x_3 
%				\end{pmatrix}
%				=
%				\begin{pmatrix}
%					k_1 \\ k_2 \\ 2k_1 - k_2 
%				\end{pmatrix}
%			\end{align*}‎
%		\end{ceqn}
%		有解.容易看出,当$k_1 = k_2$时方程有解,且其解为
%		\begin{ceqn}
%			\begin{align*}
%				-2x_1 + 2x_2 + x_3 = k_1 
%			\end{align*}‎
%		\end{ceqn}
%		其中$k_1$为任意非零常数.取$k_1=k_2=1$,可得$p_1 = \xi = \begin{pmatrix}
%			1 \\ 0 \\ 2
%		\end{pmatrix}$,$p_2 = k_1\xi + k_2\eta = \begin{pmatrix}
%			1 \\ 1 \\ 0
%		\end{pmatrix}$,取一特解$p_3 = \begin{pmatrix}
%			1 \\ 1 \\ 1
%		\end{pmatrix}$,则有$\boldsymbol{P} = \begin{pmatrix}
%			1 & 1 & 1 \\ 0 & 1 & 1 \\ 2 & 0 & 1
%		\end{pmatrix}$使得$\boldsymbol{P}^{-1}\boldsymbol{AP} = \boldsymbol{J} = 
%		\begin{pmatrix}
%			4 & 0 & 0 \\
%			0 & 4 & 1 \\
%			0 & 0 & 4
%		\end{pmatrix}$.\\[12pt]
%		\indent 矩阵函数$f(\boldsymbol{A}) = \boldsymbol{P}^{-1}
%		\begin{pmatrix}
%			f(\boldsymbol{J_1}) & \\ & f(\boldsymbol{J_2})
%		\end{pmatrix}
%		\boldsymbol{P}$,\\[6pt]			%这个[6pt]是在普通行距的基础上加的
%		其中$\boldsymbol{P} = \begin{pmatrix}
%			1 & 1 & 1 \\ 0 & 1 & 1 \\ 2 & 0 & 1
%		\end{pmatrix} \text{,可求得}
%		\boldsymbol{P}^{-1} = \begin{pmatrix}
%			1 & -1 & 0 \\ 2 & -1 & -1 \\ -2 & 2 & 1
%		\end{pmatrix} \text{,}\\
%		f(\boldsymbol{J_i}) = 
%		\begin{pmatrix}
%			f(\lambda_i) & \dfrac{1}{1!}f'(\lambda_i) & \cdots & \dfrac{1}{(n_i-1)!}f^{(n_i-1)}(\lambda_i) \\
%			& f(\lambda_i) & \ddots & \vdots \\
%			& & \ddots & \dfrac{1}{1!}f'(\lambda_i) \\
%			& & & f(\lambda_i)
%		\end{pmatrix}(i = 1,2; \  n_i \text{为}\boldsymbol{J_i}\text{的阶数})$ \\
%		则有
%		\begin{ceqn}
%			\begin{gather*}
%				f(\boldsymbol{J_1}) = \begin{pmatrix} f(\lambda_1) \end{pmatrix} \\
%				f(\boldsymbol{J_2}) = \begin{pmatrix} f(\lambda_2) & f'(\lambda_2) \\
%					& f(\lambda_2)
%				\end{pmatrix} 
%			\end{gather*}
%		\end{ceqn}
%		$f(x) = e^x$时,$f'(x) = e^x$ ,$f(\lambda_1) = e^4$,$f(\lambda_2) = e^4$,$f'(\lambda_2) = e^4$
%		\begin{ceqn}
%			\begin{align*}
%			f(\boldsymbol{A}) &= 
%			\boldsymbol{P}^{-1}
%			\begin{pmatrix}
%				f(\boldsymbol{J_1}) & \\ & f(\boldsymbol{J_2})
%			\end{pmatrix}
%			\boldsymbol{P}
%			=
%			\begin{pmatrix}
%				1 & -1 & 0 \\ 2 & -1 & -1 \\ -2 & 2 & 1
%			\end{pmatrix}
%			\begin{pmatrix}
%			e^4 &  & \\
%			& e^4 & e^4 \\
%			& & e^4
%			\end{pmatrix}
%			\begin{pmatrix}
%				1 & 1 & 1 \\ 0 & 1 & 1 \\ 2 & 0 & 1
%			\end{pmatrix} \\
%			&=
%			\begin{pmatrix}
%				e^4 & -e^4 & -e^4 \\
%				2e^4 & -e^4 & -2e^4 \\
%				-2e^4 & 2e^4 & 3e^4
%			\end{pmatrix}
%			\end{align*}
%		\end{ceqn}
%		$f(x) = e^{tx}$时,$f'(x) = te^{tx}$ ,$f(\lambda_1) = e^{4t}$,$f(\lambda_2) = e^{4t}$,$f'(\lambda_2) = te^{4t}$
%		$$f(\boldsymbol{A}) = 
%		\begin{pmatrix}
%			e^{4t} &  & \\
%			& e^{4t} & te^{4t} \\
%			& & e^{4t}
%		\end{pmatrix}
%		$$
%		$f(x) = sin{tx}$时,$f'(x) = tcos{tx}$ ,$f(\lambda_1) = sin4t$,$f(\lambda_2) = sin4t$,$f'(\lambda_2) = tcos4t$
%		$$f(\boldsymbol{A}) = 
%		\begin{pmatrix}
%			sin4t &  & \\
%			& sin4t & tcos4t \\
%			& & sin4t
%		\end{pmatrix}
%		$$
%		\\
		
		
		
		\subsection*{\kaishu {(4)解:}}
		$$
		\boldsymbol{A} = 
		\begin{pmatrix}
			2 & 0 & 0 \\ 1 & 1 & 1 \\ 1 & -1 & 3
		\end{pmatrix}
		$$
		有
		\begin{gather*}
			\lambda \boldsymbol{I} - \boldsymbol{A} = 
			\begin{pmatrix}
				\lambda-2 & 0 & 0 \\
				-1 & \lambda-1 & -1 \\
				-1 & 1 & \lambda-3
			\end{pmatrix}
			\to
			\begin{pmatrix}
				1 & -1 & -\lambda+3 \\
				-1 & \lambda-1 & -1 \\
				\lambda-2 & 0 & 0
			\end{pmatrix}	\\
			\to
			\begin{pmatrix}
				-1 & -1 & -\lambda+3 \\
				0 & \lambda-2 & -\lambda+2 \\
				0 & \lambda-2 & (\lambda-2)(\lambda-3)
			\end{pmatrix}
			\to
			\begin{pmatrix}
				1 & 0 & 0 \\
				0 & \lambda-2 & -(\lambda-2) \\
				0 & \lambda-2 & (\lambda-2)(\lambda-3)
			\end{pmatrix}	\\
			\to
			\begin{pmatrix}
				1 &  &  \\
				& \lambda-2 &  \\
				&  & (\lambda-2)^2
			\end{pmatrix}
		\end{gather*}
		\indent 则$\boldsymbol{A}$的初等因子为$\lambda-2$,$(\lambda-2)^2$,故$\boldsymbol{A}$的$Jordan$标准形为\par
		\begin{ceqn}
			\begin{align*}
				\boldsymbol{J} =
				\begin{pmatrix}
					2 & 0 & 0 \\
					0 & 2 & 1 \\
					0 & 0 & 2
				\end{pmatrix} ‎
			\end{align*}‎
		\end{ceqn}
		其中Jordan块
		$$
		\begin{array}{cc}
			\boldsymbol{J_1} = \begin{pmatrix} 2 \end{pmatrix} \quad & \quad
			\boldsymbol{J_2} = \begin{pmatrix} 2 & 1 \\ 0 & 2 \end{pmatrix}
		\end{array}
		$$
		则$\lambda_1 = \lambda_2 = 2$. \\[12pt]
		\indent 求变换矩阵$\boldsymbol{P}$使得$\boldsymbol{P}^{-1}\boldsymbol{AP}=\boldsymbol{J}$,即求$\boldsymbol{P}=(p_1,p_2,p_3)$满足$\boldsymbol{AP}=\boldsymbol{PJ}$,带入$\boldsymbol{A}$、$\boldsymbol{P}$、$\boldsymbol{J}$得:
		\begin{ceqn}
			\begin{align*}
				(\boldsymbol{A}p_1,\boldsymbol{A}p_2,\boldsymbol{A}p_3) = (p_1,p_2,p_3)
				\begin{pmatrix}
					2 & 0 & 0  \\
					0 & 2 & 1  \\
					0 & 0 & 2  
				\end{pmatrix} ‎
			\end{align*}‎
		\end{ceqn}
		\indent 比较上式两边得
		\begin{ceqn}
			\begin{align*}
				\begin{cases}
					\boldsymbol{A}p_1 = 2p_1,\\
					\boldsymbol{A}p_2 = 2p_2,\\
					\boldsymbol{A}p_3 = p_2 + 2p_3.
				\end{cases} 
			\end{align*}‎
		\end{ceqn}
		由此可见,$p_1$,$p_2$是$\boldsymbol{A}$的对应于特征值$2$的两个线性无关的特征向量.\par
		\indent 从方程组
		\begin{ceqn}
			\begin{align*}
				(\boldsymbol{A} - 2\boldsymbol{I}) x = 0 
			\end{align*}‎
		\end{ceqn}
		可求得两个线性无关的特征向量$\xi = \begin{pmatrix}
			1 \\ 0 \\ -1
		\end{pmatrix}$,$\eta = \begin{pmatrix}
			0 \\ 1 \\ 1
		\end{pmatrix}$. \\
		\indent 取$p_1 = \xi$,$p_2 = k_1\xi + k_2\eta$,其满足方程 $\boldsymbol{A}p_3 = p_2 + 2p_3$ 有解,记$p_3 = (x_1,x_2,x_3)^T$,即
		\begin{ceqn}
			\begin{align*}
				\begin{pmatrix}
					0 & 0 & 0 \\
					-1 & 1 & -1 \\
					-1 & 1 & -1
				\end{pmatrix}
				\begin{pmatrix}
					x_1 \\ x_2 \\ x_3 
				\end{pmatrix}
				=
				\begin{pmatrix}
					k_1 \\ k_2 \\ -k_1 + k_2 
				\end{pmatrix}
			\end{align*}‎
		\end{ceqn}
		有解.容易看出,当$k_1 =0$时方程有解,且其解为
		\begin{ceqn}
			\begin{align*}
				-2x_1 + 2x_2 + x_3 = k_2 
			\end{align*}‎
		\end{ceqn}
		其中$k_2$为任意非零常数.取$k_2=1$,可得$p_1 = \xi = \begin{pmatrix}
			1 \\ 0 \\ -1
		\end{pmatrix}$,$p_2 = k_1\xi + k_2\eta = \begin{pmatrix}
			0 \\ 1 \\ 1
		\end{pmatrix}$,取一特解$p_3 = \begin{pmatrix}
			0 \\ -1 \\ 0
		\end{pmatrix}$,则有$\boldsymbol{P} = \begin{pmatrix}
			1 & 0 & 0 \\ 0 & 1 & -1 \\ -1 & 1 & 0
		\end{pmatrix}$使得$\boldsymbol{P}^{-1}\boldsymbol{AP} = \boldsymbol{J} = 
		\begin{pmatrix}
			2 & 0 & 0 \\
			0 & 2 & 1 \\
			0 & 0 & 2
		\end{pmatrix}$.\\[12pt]
		\indent 矩阵函数$f(\boldsymbol{A}) = \boldsymbol{P}
		\begin{pmatrix}
			f(\boldsymbol{J_1}) & \\ & f(\boldsymbol{J_2})
		\end{pmatrix}
		\boldsymbol{P}^{-1}$,\\[6pt]			%这个[6pt]是在普通行距的基础上加的
		其中$\boldsymbol{P} = \begin{pmatrix}
			1 & 0 & 0 \\ 0 & 1 & -1 \\ -1 & 1 & 0
		\end{pmatrix} \text{,可求得}
		\boldsymbol{P}^{-1} = \begin{pmatrix}
			1 & 0 & 0 \\ 1 & 0 & 1 \\ 1 & -1 & 1
		\end{pmatrix} \text{,}\\
		f(\boldsymbol{J_i}) = 
		\begin{pmatrix}
			f(\lambda_i) & \dfrac{1}{1!}f'(\lambda_i) & \cdots & \dfrac{1}{(n_i-1)!}f^{(n_i-1)}(\lambda_i) \\
			& f(\lambda_i) & \ddots & \vdots \\
			& & \ddots & \dfrac{1}{1!}f'(\lambda_i) \\
			& & & f(\lambda_i)
		\end{pmatrix}(i = 1,2; \  n_i \text{为}\boldsymbol{J_i}\text{的阶数})$ \\
		则有
		\begin{ceqn}
			\begin{gather*}
				f(\boldsymbol{J_1}) = \begin{pmatrix} f(\lambda_1) \end{pmatrix} \\
				f(\boldsymbol{J_2}) = \begin{pmatrix} f(\lambda_2) & f'(\lambda_2) \\
					& f(\lambda_2)
				\end{pmatrix} 
			\end{gather*}
		\end{ceqn}
		$f(x) = e^x$时,$f'(x) = e^x$ ,$f(\lambda_1) = e^4$,$f(\lambda_2) = e^4$,$f'(\lambda_2) = e^4$
		\begin{ceqn}
			\begin{align*}
				f(\boldsymbol{A}) &= 
				\boldsymbol{P}
				\begin{pmatrix}
					f(\boldsymbol{J_1}) & \\ & f(\boldsymbol{J_2})
				\end{pmatrix}
				\boldsymbol{P}^{-1}
				=
				\begin{pmatrix}
					1 & 0 & 0 \\ 0 & 1 & -1 \\ -1 & 1 & 0
				\end{pmatrix}
				\begin{pmatrix}
					e^2 &  & \\
					& e^2 & e^2 \\
					& & e^2
				\end{pmatrix}
				\begin{pmatrix}
					1 & 0 & 0 \\ 1 & 0 & 1 \\ 1 & -1 & 1
				\end{pmatrix} \\
				&=
				\begin{pmatrix}
					e^2 & 0 & 0 \\
					e^2 & 0 & e^2 \\
					e^2 & -e^2 & 2e^2
				\end{pmatrix}
			\end{align*}
		\end{ceqn}
		$f(x) = e^{tx}$时,$f'(x) = te^{tx}$ ,$f(\lambda_1) = e^{2t}$,$f(\lambda_2) = e^{2t}$,$f'(\lambda_2) = te^{2t}$
		\begin{ceqn}
			\begin{align*}
				f(\boldsymbol{A}) &= 
				\boldsymbol{P}
				\begin{pmatrix}
					f(\boldsymbol{J_1}) & \\ & f(\boldsymbol{J_2})
				\end{pmatrix}
				\boldsymbol{P}^{-1}
				=
				\begin{pmatrix}
					1 & 0 & 0 \\ 1 & 0 & 1 \\ 1 & -1 & 1
				\end{pmatrix}
				\begin{pmatrix}
					e^{2t} &  & \\
					& e^{2t} & te^{2t} \\
					& & e^{2t}
				\end{pmatrix}
				\begin{pmatrix}
					1 & 0 & 0 \\ 0 & 1 & -1 \\ -1 & 1 & 0
				\end{pmatrix} \\
				&=
				\begin{pmatrix}
					e^{2t} & 0 & 0 \\
					te^{2t} & (1-t)e^{2t} & te^{2t} \\
					te^{2t} & -te^{2t} & (1+t)e^{2t}
				\end{pmatrix}
			\end{align*}
		\end{ceqn}
		$f(x) = sin{tx}$时,$f'(x) = tcos{tx}$ ,$f(\lambda_1) = sin2t$,$f(\lambda_2) = sin2t$,$f'(\lambda_2) = tcos2t$
		\begin{ceqn}
			\begin{align*}
				f(\boldsymbol{A}) &= 
				\boldsymbol{P}
				\begin{pmatrix}
					f(\boldsymbol{J_1}) & \\ & f(\boldsymbol{J_2})
				\end{pmatrix}
				\boldsymbol{P}^{-1}
				=
				\begin{pmatrix}
					1 & 0 & 0 \\ 1 & 0 & 1 \\ 1 & -1 & 1
				\end{pmatrix}
				\begin{pmatrix}
					sin2t &  & \\
					& sin2t & tcos2t \\
					& & sin2t
				\end{pmatrix}
				\begin{pmatrix}
					1 & 0 & 0 \\ 0 & 1 & -1 \\ -1 & 1 & 0
				\end{pmatrix} \\
				&=
				\begin{pmatrix}
					sin2t & 0 & 0 \\
					tcos2t & sin2t-tcos2t & tcos2t \\
					tcos2t & -tcos2t & sin2t+tcos2t
				\end{pmatrix}
			\end{align*}
		\end{ceqn}
		\clearpage
		
		
		
	\section*{\kaishu 第八章-习题15.}
		\subsection*{\kaishu {(1)解:}}
		$\text{原方程组的系数矩阵}\ \boldsymbol{A} = 
		\begin{pmatrix}
			1 & 2 & 3 & -1 \\
			3 & 2 & 1 & -1 \\
			2 & 3 & 1 & 1
		\end{pmatrix}
		\text{,常数项矩阵}\ b = 
		\begin{pmatrix}
			1 \\ 1 \\ 1
		\end{pmatrix}$. \\
		取$\boldsymbol{L_1} = 
		\begin{pmatrix}
			1 & & \\
			-3 & 1 & 1 \\ 
			-2 & & 1
		\end{pmatrix}$,
		有$\boldsymbol{L_1}\boldsymbol{A} = 
		\begin{pmatrix}
			1 & 2 & 3 & -1 \\
			0 & -4 & -8 & 2 \\
			0 & -1 & -5 & 3
		\end{pmatrix} =
		\boldsymbol{A}^{(1)}$.
		显然$\boldsymbol{A}$为行满秩矩阵. 故
		\begin{ceqn}
			\begin{align*}
				\boldsymbol{A}^{+} = \boldsymbol{A}^T(\boldsymbol{A}\boldsymbol{A}^T)^{-1} &= 
				\begin{pmatrix}
					1 & 3 & 2 \\
					2 & 2 & 3 \\
					3 & 1 & 1 \\
					-1 & -1 & 1
				\end{pmatrix}
				{\left[
				\begin{pmatrix}
					1 & 2 & 3 & -1 \\
					3 & 2 & 1 & -1 \\
					2 & 3 & 1 & 1
				\end{pmatrix}
				\begin{pmatrix}
					1 & 3 & 2 \\
					2 & 2 & 3 \\
					3 & 1 & 1 \\
					-1 & -1 & 1
				\end{pmatrix}
				\right]}^{-1}  \\
				&= \dfrac{1}{270}
				\begin{pmatrix}
					-45 & 95 & -10 \\
					9 & -25 & 68 \\
					90 & -40 & -10 \\
					-27 & -75 & 96 
				\end{pmatrix}
			\end{align*}
		\end{ceqn}
		有$\boldsymbol{A}\boldsymbol{A}^{+} = 
		\begin{pmatrix}
			1 & 2 & 3 & -1 \\
			3 & 2 & 1 & -1 \\
			2 & 3 & 1 & 1
		\end{pmatrix}
		\times
		\dfrac{1}{270}
		\begin{pmatrix}
			-45 & 95 & -10 \\
			9 & -25 & 68 \\
			90 & -40 & -10 \\
			-27 & -75 & 96 
		\end{pmatrix} = 
		\begin{pmatrix}
			1 & 0 & 0 \\
			0 & 1 & 0 \\
			0 & 0 & 1
		\end{pmatrix}$ \\[8pt]
		$\boldsymbol{A}\boldsymbol{A}^{+}b = b$,故原方程组相容,其通解为:
		\begin{ceqn}
			\begin{align*}
				x &= \boldsymbol{A}^{+}b + (\boldsymbol{I} - \boldsymbol{A}^{+}\boldsymbol{A})y \\
				&= \dfrac{1}{270}
				\begin{pmatrix}
					-45 & 95 & -10 \\
					9 & -25 & 68 \\
					90 & -40 & -10 \\
					-27 & -75 & 96 
				\end{pmatrix}
				\begin{pmatrix}
					1 \\ 1 \\ 1
				\end{pmatrix}
				+
				\left[
				\begin{pmatrix}
					1 & & & \\
					 & 1 & & \\
					 & & 1 & \\
					 & & & 1
				\end{pmatrix}
				-
				\dfrac{1}{270}
				\begin{pmatrix}
					-45 & 95 & -10 \\
					9 & -25 & 68 \\
					90 & -40 & -10 \\
					-27 & -75 & 96 
				\end{pmatrix}
				\begin{pmatrix}
					1 & 2 & 3 & -1 \\
					3 & 2 & 1 & -1 \\
					2 & 3 & 1 & 1
				\end{pmatrix}
				\right] y \\
				&=
				\frac{1}{270}
				\begin{pmatrix}
					40 \\ 52 \\ 40 \\ -6
				\end{pmatrix}
				+
				\left[
				\begin{pmatrix}
					1 & & & \\
					& 1 & & \\
					& & 1 & \\
					& & & 1
				\end{pmatrix}
				-
				\dfrac{1}{270}
				\begin{pmatrix}
					220 & 70 & -50 & -60 \\
					70 & 172 & 70 & 84 \\
					-50 & 70 & 220 & -60 \\
					-60 & 84 & -60 & 198
				\end{pmatrix}
				\right] y \\
				&= 
				\frac{1}{270}
				\begin{pmatrix}
					40 \\ 52 \\ 40 \\ -6
				\end{pmatrix}
				+
				\frac{1}{270}
				\begin{pmatrix}
					50 & -70 & 50 & 60 \\
					-70 & 98 & -70 & -84 \\
					50 & -70 & 50 & 60 \\
					60 & -84 & 60 & 72
				\end{pmatrix} y
				\text{,其中}y\in\mathbb{R}
			 \end{align*}
		 \end{ceqn}
		其极小范数解为:
		\begin{ceqn}
			\begin{align*}
				x = \boldsymbol{A}^{+}b 
				= \frac{1}{270}
				\begin{pmatrix}
					40 \\ 52 \\ 40 \\ -6
				\end{pmatrix}
			\end{align*}
		\end{ceqn}
		\\\\
		
		
		\subsection*{\kaishu {(2)解:}}
		$\text{原方程组的系数矩阵}\ \boldsymbol{A} = 
		\begin{pmatrix}
			1 & 1 & 0 \\
			1 & 0 & 1 \\
			-1 & 0 & 0 \\
			 1 & 1 & 1
		\end{pmatrix}
		\text{,常数项矩阵}\ b = 
		\begin{pmatrix}
			0 \\ 0 \\ 1 \\ 2
		\end{pmatrix}$. \\
		取$\boldsymbol{L_1} = 
		\begin{pmatrix}
			1 & & & \\
			-1 & 1 & & \\ 
			1 & & 1 & \\
			-1 & & & 1
		\end{pmatrix}$,
		有$\boldsymbol{L_1}\boldsymbol{A} = 
		\begin{pmatrix}
			1 & 1 & 0 \\
			0 & -1 & 1 \\
			0 & 1 & 0 \\
			0 & 0 & 1
		\end{pmatrix} =
		\boldsymbol{A}^{(1)}$.\\
		取$\boldsymbol{L_2} = 
		\begin{pmatrix}
			1 & & & \\
			 & 1 & & \\ 
			 & 1 & 1 & \\
			 & & & 1
		\end{pmatrix}$,
		有$\boldsymbol{L_2}\boldsymbol{L_1}\boldsymbol{A}^{(1)} = 
		\begin{pmatrix}
			1 & 1 & 0 \\
			0 & -1 & 1 \\
			0 & 0 & 1 \\
			0 & 0 & 1
		\end{pmatrix} =
		\boldsymbol{A}^{(2)}$.\\
		取$\boldsymbol{L_3} = 
		\begin{pmatrix}
			1 & & & \\
			 & 1 & & \\ 
			 & 1 & 1 & \\
			 & & -1 & 1
		\end{pmatrix}$,
		有$\boldsymbol{L_3}\boldsymbol{L_2}\boldsymbol{L_1}\boldsymbol{A}^{(1)} = 
		\begin{pmatrix}
			1 & 1 & 0 \\
			0 & -1 & 1 \\
			0 & 0 & 1 \\
			0 & 0 & 0
		\end{pmatrix} =
		\boldsymbol{A}^{(3)}$.\\[8pt]
		显然$\boldsymbol{A}$为列满秩矩阵. 故
		\begin{ceqn}
			\begin{align*}
				\boldsymbol{A}^{+} = (\boldsymbol{A}^T\boldsymbol{A})^{-1}\boldsymbol{A}^T &= 
				{\left[
					\begin{pmatrix}
						1 & 1 & -1 & 1 \\
						1 & 0 & 0 & 1 \\
						0 & 1 & 0 & 1
					\end{pmatrix}
					\begin{pmatrix}
						1 & 1 & 0 \\
						1 & 0 & 1 \\
						-1 & 0 & 0 \\
						1 & 1 & 1
					\end{pmatrix}
					\right]}^{-1}
					\begin{pmatrix}
						1 & 1 & -1 & 1 \\
						1 & 0 & 0 & 1 \\
						0 & 1 & 0 & 1
					\end{pmatrix}  \\
				&= \dfrac{1}{4}
				\begin{pmatrix}
					1 & 1 & -3 & -1 \\
					2 & -2 & 2 & 2 \\
					-2 & 2 & 2 & 2 
				\end{pmatrix}
			\end{align*}
		\end{ceqn}
		有$\boldsymbol{A}\boldsymbol{A}^{+} = 
		\begin{pmatrix}
			1 & 1 & 0 \\
			1 & 0 & 1 \\
			-1 & 0 & 0 \\
			1 & 1 & 1
		\end{pmatrix}
		\times
		\dfrac{1}{4}
		\begin{pmatrix}
			1 & 1 & -3 & -1 \\
			2 & -2 & 2 & 2 \\
			-2 & 2 & 2 & 2
		\end{pmatrix} = 
		\dfrac{1}{4}
		\begin{pmatrix}
			3 & -1 & -1 & 1 \\
			-1 & 3 & -1 & 1 \\
			-1 & -1 & 3 & 1 \\
			1 & 1 & 1 & 3
		\end{pmatrix}$ \\[8pt]
		$\boldsymbol{A}\boldsymbol{A}^{+}b = 
		\dfrac{1}{4}
		\begin{pmatrix}
			1 \\ 1 \\ 5 \\ 7
		\end{pmatrix}
		\ne b = \begin{pmatrix}
			0 \\ 0 \\ 1 \\ 2
		\end{pmatrix}$,故原方程组不相容,其最小二乘通解为:
		\begin{ceqn}
			\begin{align*}
				x &= \boldsymbol{A}^{+}b + (\boldsymbol{I} - \boldsymbol{A}^{+}\boldsymbol{A})y \\
				&= \dfrac{1}{4}
				\begin{pmatrix}
					1 & 1 & -3 & -1 \\
					2 & -2 & 2 & 2 \\
					-2 & 2 & 2 & 2
				\end{pmatrix}
				\begin{pmatrix}
					0 \\ 0 \\ 1 \\ 2
				\end{pmatrix}
				+
				\left[
				\begin{pmatrix}
					1 & & \\
					& 1 & \\
					& & 1 
				\end{pmatrix}
				-
				\dfrac{1}{4}
				\begin{pmatrix}
					1 & 1 & -3 & -1 \\
					2 & -2 & 2 & 2 \\
					-2 & 2 & 2 & 2
				\end{pmatrix}
				\begin{pmatrix}
					1 & 1 & 0 \\
					1 & 0 & 1 \\
					-1 & 0 & 0 \\
					1 & 1 & 1
				\end{pmatrix}
				\right] y \\
				&=
				\frac{1}{4}
				\begin{pmatrix}
					-5 \\ 6 \\ 6
				\end{pmatrix}
				+
				\boldsymbol{0} \cdot y = 
				\frac{1}{4}
				\begin{pmatrix}
					-5 \\ 6 \\ 6
				\end{pmatrix}
			\end{align*}
		\end{ceqn}
		其极小范数解为:
		\begin{ceqn}
			\begin{align*}
				x = \boldsymbol{A}^{+}b 
				= \frac{1}{4}
				\begin{pmatrix}
					-5 \\ 6 \\ 6
				\end{pmatrix}
			\end{align*}
		\end{ceqn}





















\end{document}