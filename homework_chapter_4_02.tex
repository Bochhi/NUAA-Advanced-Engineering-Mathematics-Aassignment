\documentclass[12pt,fleqn]{article}%全局的默认对齐方式设置为左对齐。
%\usepackage[fleqn]{amsmath}默认是左对齐,而且是全局设置,所以之后对于段内公式即使加入\centering或者是都无法做到使公式居中。
\usepackage{ctex}
\usepackage{amsmath}
\usepackage{geometry}
\usepackage{nccmath}
\usepackage{amssymb}
%\usepackage{bm}
%\usepackage{multicol} %用于实现在同一页中实现不同的分栏
\usepackage{color}


\linespread{}

\geometry{a4paper,left=1in, right=1in, top=1in, bottom=1in}

\title{\heiti 矩阵论第四章作业}
\author{XXX SZ2305XXX}
\date{2023年10月13日}

\renewcommand\arraystretch{0.8}
\renewcommand{\baselinestretch}{1.5}


%\newenvironment{amatrix}[1]{%
%	\left(\begin{array}{@{}*{#1}{c}|c@{}}
%	}{%
%	\end{array}\right)}




\begin{document}
	\maketitle
	\section*{\kaishu 习题3.}
	\subsection*{(1) \kaishu {解:}}
	\begin{ceqn}
		\begin{align*}
			\boldsymbol{A} =
			\begin{pmatrix}
				1 & 2 & 3 & 0 \\
				0 & 2 & 1 & -1 \\
				1 & 0 & 2 & 1 
			\end{pmatrix} 
		\end{align*}‎
	\end{ceqn}
	取
	\begin{ceqn}
		\begin{align*}
			\boldsymbol{L}_1 =
			\begin{pmatrix}
				1 & 0 & 0 \\
				0 & 1 & 0 \\
				-1 & 0 & 1 
			\end{pmatrix} 
		\end{align*}‎
	\end{ceqn}
	则
	\begin{ceqn}
		\begin{align*}
		\boldsymbol{L}_1\boldsymbol{A} =
			\begin{pmatrix}
				1 & 2 & 3 & 0 \\
				0 & 2 & 1 & -1 \\
				0 & -2 & -1 & 1
			\end{pmatrix} 
			= \boldsymbol{A} ^ {(1)}
		\end{align*}‎
	\end{ceqn}
	取
	\begin{ceqn}
		\begin{align*}
			\boldsymbol{L}_2 =
			\begin{pmatrix}
				1 & 0 & 0 \\
				0 & 1 & 0 \\
				0 & 1 & 1 
			\end{pmatrix} 
		\end{align*}‎
	\end{ceqn}
	则
	\begin{ceqn}
		\begin{align*}
			\boldsymbol{L}_2(\boldsymbol{L}_1\boldsymbol{A}) =
			\begin{pmatrix}
				1 & 2 & 3 & 0 \\
				0 & 2 & 1 & -1 \\
				0 & 0 & 0 & 0
			\end{pmatrix} 
			= \boldsymbol{A} ^ {(2)}
		\end{align*}‎
	\end{ceqn}
	因此
	\begin{ceqn}
		\begin{align*}
			\boldsymbol{A} = \boldsymbol{L}_1^{-1} \boldsymbol{L}_2^{-1} \boldsymbol{A} ^ {(2)} &=
			\begin{pmatrix}
				1 & 0 & 0 \\
				0 & 1 & 0 \\
				1 & 0 & 1 
			\end{pmatrix} 
			\begin{pmatrix}
				1 & 0 & 0 \\
				0 & 1 & 0 \\
				0 & -1 & 1 
			\end{pmatrix} 
			\begin{pmatrix}
				1 & 2 & 3 & 0 \\
				0 & 2 & 1 & -1 \\
				0 & 0 & 0 & 0
			\end{pmatrix}  \\
			&= 
			\begin{pmatrix}
				1 & 0 & 0 \\
				0 & 1 & 0 \\
				1 & -1 & 1 
			\end{pmatrix} 
			\begin{pmatrix}
				1 & 2 & 3 & 0 \\
				0 & 2 & 1 & -1 \\
				0 & 0 & 0 & 0
			\end{pmatrix} \\
			&= 
			\begin{pmatrix}
				1 & 0 \\
				0 & 1 \\
				1 & -1 
			\end{pmatrix} 
			\begin{pmatrix}
				1 & 2 & 3 & 0 \\
				0 & 2 & 1 & -1 
			\end{pmatrix}
		\end{align*}‎
	\end{ceqn}
	令$\boldsymbol{B} = \begin{pmatrix}
		1 & 0 \\
		0 & 1 \\
		1 & -1  
	\end{pmatrix} $,
	$\boldsymbol{C} = \begin{pmatrix}
		1 & 2 & 3 & 0 \\
		0 & 2 & 1 & -1 
	\end{pmatrix}$,
	则$\boldsymbol{A} = \boldsymbol{BC}$.
	此即$\boldsymbol{A}$的一个满秩分解.\\ \\ \\

	
	
	\subsection*{(4) \kaishu {解:}}
	\begin{ceqn}
		\begin{align*}
			\boldsymbol{A} =
			\begin{pmatrix}
				1 & 1 & 1 & 1 & 1 \\
				3 & 2 & 1 & 1 & -3 \\
				0 & 1 & 2 & 2 & 6 \\
				5 & 4 & 3 & 3 & -1 
			\end{pmatrix} 
		\end{align*}‎
	\end{ceqn}
	取
	\begin{ceqn}
		\begin{align*}
			\boldsymbol{L}_1 =
			\begin{pmatrix}
				1 & 0 & 0 & 0 \\
				-3 & 1 & 0 & 0 \\
				0 & 0 & 1 & 0 \\
				-5 & 0 & 0 & 1 
			\end{pmatrix} 
		\end{align*}‎
	\end{ceqn}
	则
	\begin{ceqn}
		\begin{align*}
			\boldsymbol{L}_1\boldsymbol{A} =
			\begin{pmatrix}
				1 & 1 & 1 & 1 & 1 \\
				0 & -1 & -2 & -2 & -6 \\
				0 & 1 & 2 & 2 & 6 \\
				0 & -1 & -2 & -2 & -6 
			\end{pmatrix} 
			= \boldsymbol{A} ^ {(1)}
		\end{align*}‎
	\end{ceqn}
	取
	\begin{ceqn}
		\begin{align*}
			\boldsymbol{L}_2 =
			\begin{pmatrix}
				1 & 0 & 0 & 0 \\
				0 & -1 & 0 & 0 \\
				0 & 1 & 1 & 0 \\
				0 & -1 & 0 & 1 
			\end{pmatrix} 
		\end{align*}‎
	\end{ceqn}
	则
	\begin{ceqn}
		\begin{align*}
			\boldsymbol{L}_2(\boldsymbol{L}_1\boldsymbol{A}) =
			\begin{pmatrix}
				1 & 1 & 1 & 1 & 1 \\
				0 & 1 & 2 & 2 & 6 \\
				0 & 0 & 0 & 0 & 0 \\
				0 & 0 & 0 & 0 & 0 
			\end{pmatrix} 
			= \boldsymbol{A} ^ {(2)}
		\end{align*}‎
	\end{ceqn}
	因此
	\begin{ceqn}
		\begin{align*}
			\boldsymbol{A} = \boldsymbol{L}_1^{-1} \boldsymbol{L}_2^{-1} \boldsymbol{A} ^ {(2)} &=
			\begin{pmatrix}
				1 & 0 & 0 & 0 \\
				3 & 1 & 0 & 0 \\
				0 & 0 & 1 & 0 \\
				5 & 0 & 0 & 1 
			\end{pmatrix} 
			\begin{pmatrix}
				1 & 0 & 0 & 0 \\
				0 & -1 & 0 & 0 \\
				0 & 1 & 1 & 0 \\
				0 & -1 & 0 & 1 
			\end{pmatrix} 
			\begin{pmatrix}
				1 & 1 & 1 & 1 & 1 \\
				0 & 1 & 2 & 2 & 6 \\
				0 & 0 & 0 & 0 & 0 \\
				0 & 0 & 0 & 0 & 0 
			\end{pmatrix}  \\
			&= 
			\begin{pmatrix}
				1 & 0 & 0 & 0 \\
				3 & -1 & 0 & 0 \\
				0 & 1 & 1 & 0 \\
				5 & -1 & 0 & 1 
			\end{pmatrix} 
			\begin{pmatrix}
				1 & 1 & 1 & 1 & 1 \\
				0 & 1 & 2 & 2 & 6 \\
				0 & 0 & 0 & 0 & 0 \\
				0 & 0 & 0 & 0 & 0 
			\end{pmatrix} \\
			&= 
			\begin{pmatrix}
				1 & 0 \\
				3 & -1 \\
				0 & 1 \\
				5 & -1
			\end{pmatrix} 
			\begin{pmatrix}
				1 & 1 & 1 & 1 & 1 \\
				0 & 1 & 2 & 2 & 6
			\end{pmatrix}
		\end{align*}‎
	\end{ceqn}
	令$\boldsymbol{B} = \begin{pmatrix}
		1 & 0 \\
		3 & -1 \\
		0 & 1 \\
		5 & -1
	\end{pmatrix} $,
	$\boldsymbol{C} = \begin{pmatrix}
		1 & 1 & 1 & 1 & 1 \\
		0 & 1 & 2 & 2 & 6
	\end{pmatrix}$,
	则$\boldsymbol{A} = \boldsymbol{BC}$.
	此即$\boldsymbol{A}$的一个满秩分解.
	\clearpage
	
	
	\section*{\kaishu 习题10.}
	\subsection*{(1) \kaishu {解:}}
	%第(1)小问和第(2)小问分别用了两种书写方法。
	原方程组的系数矩阵$\boldsymbol{A}$为:
	\begin{ceqn}
		\begin{align*}
			\boldsymbol{A} =
			\begin{pmatrix}
				2 & 3 & 4 \\
				1 & 1 & 9 \\
				1 & 2 & -6 
			\end{pmatrix} 
		\end{align*}‎
	\end{ceqn}
	取
	\begin{ceqn}
		\begin{align*}
			\boldsymbol{L}_1 =
			\begin{pmatrix}
				1 & 0 & 0 \\
				-\frac12 & 1 & 0 \\
				-\frac12 & 0 & 1 
			\end{pmatrix} 
		\end{align*}‎
	\end{ceqn}
	则
	\begin{ceqn}
		\begin{align*}
			\boldsymbol{L}_1\boldsymbol{A} =
			\begin{pmatrix}
				2 & 3 & 4 \\
				0 & -\frac12 & 7 \\
				0 & \frac12 & -8 
			\end{pmatrix} 
			= \boldsymbol{A} ^ {(1)}
		\end{align*}‎
	\end{ceqn}
	取
	\begin{ceqn}
		\begin{align*}
			\boldsymbol{L}_2 =
			\begin{pmatrix}
				1 & 0 & 0 \\
				0 & 1 & 0 \\
				0 & 1 & 1 
			\end{pmatrix} 
		\end{align*}‎
	\end{ceqn}
	则
	\begin{ceqn}
		\begin{align*}
			\boldsymbol{L}_2(\boldsymbol{L}_1\boldsymbol{A}) =
			\begin{pmatrix}
				2 & 3 & 4 \\
				0 & -\frac12 & 7 \\
				0 & 0 & -1 
			\end{pmatrix} 
			= \boldsymbol{A} ^ {(2)}
		\end{align*}‎
	\end{ceqn}
	因此
	\begin{ceqn}
		\begin{align*}
			\boldsymbol{A} = \boldsymbol{L}_1^{-1} \boldsymbol{L}_2^{-1} \boldsymbol{A} ^ {(2)} &=
			\begin{pmatrix}
				1 & 0 & 0 \\
				\frac12 & 1 & 0 \\
				\frac12 & 0 & 1 
			\end{pmatrix} 
			\begin{pmatrix}
				1 & 0 & 0 \\
				0 & 1 & 0 \\
				0 & -1 & 1
			\end{pmatrix}
			\begin{pmatrix}
				2 & 3 & 4 \\
				0 & -\frac12 & 7 \\
				0 & 0 & -1 
			\end{pmatrix}   \\
			&= 
			\begin{pmatrix}
				1 & 0 & 0 \\
				\frac12 & 1 & 0 \\
				\frac12 & -1 & 1 
			\end{pmatrix}
			\begin{pmatrix}
				2 & 3 & 4 \\
				0 & -\frac12 & 7 \\
				0 & 0 & -1 
			\end{pmatrix} 
		\end{align*}‎
	\end{ceqn}
	令$\boldsymbol{L} = \begin{pmatrix}
		1 & 0 & 0 \\
		\frac12 & 1 & 0 \\
		\frac12 & -1 & 1 
	\end{pmatrix} $,
	$\boldsymbol{U} = \begin{pmatrix}
		2 & 3 & 4 \\
		0 & -\frac12 & 7 \\
		0 & 0 & -1
	\end{pmatrix}$,
	则$\boldsymbol{A} = \boldsymbol{LU}$.
	此即$\boldsymbol{A}$的$\boldsymbol{LU}$分解.\\
	原方程组$\boldsymbol{A}x=b$化为$ \left\{
	\begin{aligned}
		\boldsymbol{L}y &= b \\
		\boldsymbol{U}x &= y
	\end{aligned}
	\right. $ 
	. \\
	\indent 解线性方程组
	\begin{ceqn}
		\begin{align*}
			\boldsymbol{L}y=
			\begin{pmatrix}
				1 & 0 & 0 \\
				\frac12 & 1 & 0 \\
				\frac12 & -1 & 1 
			\end{pmatrix} 
			\begin{pmatrix}
				y_1 \\
				y_2 \\
				y_3
			\end{pmatrix} 
			= 
			\begin{pmatrix}
				1 \\
				-7 \\
				9
			\end{pmatrix} 
			= b
		\end{align*}‎
	\end{ceqn}
	得$
		\begin{pmatrix}
			y_1 \\
			y_2 \\
			y_3
		\end{pmatrix}
		=
		\begin{pmatrix}
			1 \\
			-\frac{15}{2} \\
			1
		\end{pmatrix} $. \\\\
	\indent 再解线性方程组
	\begin{ceqn}
		\begin{align*}
			\boldsymbol{U}x=
			\begin{pmatrix}
				2 & 3 & 4 \\
				0 & -\frac{1}{2} & 7 \\
				0 & 0 & -1 
			\end{pmatrix} 
			\begin{pmatrix}
				x_1 \\
				x_2 \\
				x_3
			\end{pmatrix} 
			= 
			\begin{pmatrix}
				1 \\
				-\frac{15}{2} \\
				1
			\end{pmatrix} 
			= y
		\end{align*}‎
	\end{ceqn}
	得$
	\begin{pmatrix}
		x_1 \\
		x_2 \\
		x_3
	\end{pmatrix}
	=
	\begin{pmatrix}
		1 \\
		1 \\
		-1
	\end{pmatrix} $.此即原方程组的解.\\\\\\
	
	
	\subsection*{(2) \kaishu {解:}}
%	\[
%	\left(\begin{array}{ccc|ccc}  
%		2 & 3 & 4 & 1 & 0 & 0 \\  
%		3 & 5 & 2 & 0 & 1 & 0 \\
%		4 & 3 & 30 & 0 & 0 & 1   
%	\end{array}\right)
%	\]
	
	原方程组的系数矩阵$\boldsymbol{A}$为:
	\begin{ceqn}
		\begin{align*}
			\boldsymbol{A} =
			\begin{pmatrix}
				2 & 3 & 4 \\
				3 & 5 & 2 \\
				4 & 3 & 30 
			\end{pmatrix} 
		\end{align*}‎
	\end{ceqn}
	有
	\begin{ceqn}
		\begin{align*}
			(\boldsymbol{A\mid I}) =
				\left(\begin{array}{ccc|ccc}  
					2 & 3 & 4 & 1 & 0 & 0 \\  
					3 & 5 & 2 & 0 & 1 & 0 \\
					4 & 3 & 30 & 0 & 0 & 1   
				\end{array}\right)
			\rightarrow
				\left(\begin{array}{ccc|ccc}  
					2 & 3 & 4 & 1 & 0 & 0 \\  
					0 & \frac12 & -4 & -\frac32 & 1 & 0 \\
					0 & -3 & 22 & -2 & 0 & 1   
				\end{array}\right)
			\rightarrow
				\left(\begin{array}{ccc|ccc}  
					2 & 3 & 4 & 1 & 0 & 0 \\  
					0 & \frac12 & -4 & -\frac32 & 1 & 0 \\
					0 & 0 & -2 & \color{red}-11 & 6 & 1   
				\end{array}\right)
		\end{align*}‎
	\end{ceqn}
	因此$\boldsymbol A = {\color{green}
	\begin{pmatrix}
		1 & 0 & 0 \\
		\frac32 & 1 & 0 \\
		2 & -6 & 1 
	\end{pmatrix} } 
	\begin{pmatrix}
		2 & 3 & 4 \\
		0 & \frac12 & -4 \\
		0 & 0 & 2 
	\end{pmatrix} $. \qquad {\color{green} \text{(绿色矩阵是变换矩阵求逆得来的)}} \\[8pt]
	令$\boldsymbol{L} = \begin{pmatrix}
		1 & 0 & 0 \\
		\frac32 & 1 & 0 \\
		2 & -6 & 1 
	\end{pmatrix} $,
	$\boldsymbol{U} = \begin{pmatrix}
		2 & 3 & 4 \\
		0 & \frac12 & -4 \\
		0 & 0 & 2
	\end{pmatrix}$,
	则$\boldsymbol{A} = \boldsymbol{LU}$.
	此即$\boldsymbol{A}$的$\boldsymbol{LU}$分解.\\
	原方程组$\boldsymbol{A}x=b$化为$ \left\{
	\begin{aligned}
		\boldsymbol{L}y &= b \\
		\boldsymbol{U}x &= y
	\end{aligned}
	\right. $ 
	. \\
	\indent 解线性方程组
	\begin{ceqn}
		\begin{align*}
			\boldsymbol{L}y=
			\begin{pmatrix}
				1 & 0 & 0 \\
				\frac32 & 1 & 0 \\
				2 & -6 & 1
			\end{pmatrix} 
			\begin{pmatrix}
				y_1 \\
				y_2 \\
				y_3
			\end{pmatrix} 
			= 
			\begin{pmatrix}
				0 \\
				-5 \\
				28
			\end{pmatrix} 
			= b
		\end{align*}‎
	\end{ceqn}
	得$
	\begin{pmatrix}
		y_1 \\
		y_2 \\
		y_3
	\end{pmatrix}
	=
	\begin{pmatrix}
		0 \\
		-5 \\
		-2
	\end{pmatrix} $. \\\\
	\indent 再解线性方程组
	\begin{ceqn}
		\begin{align*}
			\boldsymbol{U}x=
			\begin{pmatrix}
				2 & 3 & 4 \\
				0 & \frac12 & -4 \\
				0 & 0 & 2
			\end{pmatrix} 
			\begin{pmatrix}
				x_1 \\
				x_2 \\
				x_3
			\end{pmatrix} 
			= 
			\begin{pmatrix}
				0 \\
				-5 \\
				-2
			\end{pmatrix} 
			= y
		\end{align*}‎
	\end{ceqn}
	得$
	\begin{pmatrix}
		x_1 \\
		x_2 \\
		x_3
	\end{pmatrix}
	=
	\begin{pmatrix}
		1 \\
		-2 \\
		1
	\end{pmatrix} $.此即原方程组的解.
	\clearpage
	
	
	\section*{\kaishu 习题12.}
	\subsection*{(2) \kaishu {解:}}
	\begin{ceqn}
		\begin{align*}
			\boldsymbol{A} =
			\begin{pmatrix}
				1 & 1 & -1 \\
				-1 & 1 & 1 \\
				1 & 1 & -1 \\
				1 & 1 & 1 
			\end{pmatrix}
		\end{align*}‎
	\end{ceqn}
	记$\alpha_1 = 
	\begin{pmatrix}
		1 &	-1 & 1 & 1
	\end{pmatrix}^T$,
	$\alpha_2 = 
	\begin{pmatrix}
		1 &	1 & 1 & 1
	\end{pmatrix}^T$,
	$\alpha_3 = 
	\begin{pmatrix}
		-1 & 1 & -1 & 1
	\end{pmatrix}^T$,\\[8pt]
	令
	\begin{align*}
		\beta_1 ={} & \alpha_1 = \begin{pmatrix}
			1 &	-1 & 1 & 1
		\end{pmatrix}^T, \\
		q_1 ={} & \frac{\beta_1}{{\parallel}{\beta_1}{\parallel}}
		 = %这里不打三个大括号有报错
		\frac{\alpha_1}{\sqrt{(\beta_1,\beta_1)}} 
		 = 
		 \frac{\begin{pmatrix}
		 		1 &	-1 & 1 & 1
		 	\end{pmatrix}^T}{\sqrt{1^2+1^2+1^2+1^2}}
	 	 = 
		 \frac12 \begin{pmatrix}
					1 &	-1 & 1 & 1
				\end{pmatrix}^T ,\\
		\beta_2 ={} & \alpha_2 - \frac{(\alpha_2,\beta_1)}{(\beta_1,\beta_1)}\beta_1 = 
		\begin{pmatrix}
			1 &	1 & 1 & 1
		\end{pmatrix}^T
		-
		\frac24 \begin{pmatrix}
				1 &	-1 & 1 & 1
				\end{pmatrix}^T
		 = 
		 \frac12 \begin{pmatrix}
		 	1 &	3 & 1 & 1
		 \end{pmatrix}^T, \\
		q_2 ={} & \frac{\beta_2}{{\parallel}{\beta_2}{\parallel}} = 
		\frac{\beta_2}{\sqrt{(\beta_2,\beta_2)}}  = 
		\frac{\frac12
			\begin{pmatrix}
				1 &	3 & 1 & 1
			\end{pmatrix}^T}{\sqrt{{(\frac12)}^2 (1^2+3^2+1^2+1^2)}}
		 = 
		\frac{1}{2\sqrt{3}} 
		\begin{pmatrix}
			1 &	3 & 1 & 1
		\end{pmatrix}^T ,\\
		\beta_3 ={} & \alpha_3 - \frac{(\alpha_3,\beta_1)}{(\beta_1,\beta_1)}\beta_1 - \frac{(\alpha_3,\beta_2)}{(\beta_2,\beta_2)}\beta_2 = 
		\begin{pmatrix}
			-1 & 1 & -1 & 1
		\end{pmatrix}^T
		 - 
		(\frac{-2}{4}) \times
		\begin{pmatrix}
			1 &	-1 & 1 & 1
		\end{pmatrix}^T \\
		 & - 
		\frac{\frac12 \times 2}{\frac14 \times 12} \times
		\frac12 \begin{pmatrix}
			1 &	3 & 1 & 1
		\end{pmatrix}^T
		 = 
		\frac 23 \begin{pmatrix}
			-1 & 0 & -1 & 2
		\end{pmatrix}^T, \\
		q_3 ={} & \frac{\beta_3}{{\parallel}{\beta_3}{\parallel}} = 
		\frac{\beta_3}{\sqrt{(\beta_3,\beta_3)}}  = 
		\frac{\frac 23 \begin{pmatrix}
				-1 & 0 & -1 & 2
			\end{pmatrix}^T}{\sqrt{(\frac23)^2 [(-1)^2+0^2+(-1)^2+2^2]}}
		= 
		\frac{1}{\sqrt{6}} 
		\begin{pmatrix}
			-1 & 0 & -1 & 2
		\end{pmatrix}^T.
	\end{align*}
	有
	\begin{ceqn}
	\[	\begin{cases}
			\alpha_1 = \parallel \beta_1 \parallel q_1 \\
			\alpha_2 = \parallel \beta_2 \parallel q_2 + (\alpha_2,q_1)q_1 \\
			\alpha_3 = \parallel \beta_3 \parallel q_3 + (\alpha_3,q_1)q_1 + (\alpha_3,q_2)q_2 \\
		\end{cases} \]
	\end{ceqn}
	因此,
	\begin{align*}
		\boldsymbol{A} &= 
		\begin{pmatrix}
			\alpha_1 & \alpha_2 & \alpha_3
		\end{pmatrix}
		= 
		\begin{pmatrix}
			q_1 & q_2 & q_3
		\end{pmatrix}
		\begin{pmatrix}
			{\parallel}{\beta_1}{\parallel} & (\alpha_2,q_1) & (\alpha_3,q_1) \\
			0 & {\parallel}{\beta_2}{\parallel} & (\alpha_2,q_2) \\
			0 & 0 & {\parallel}{\beta_3}{\parallel} 
		\end{pmatrix} \\
		&= 
		\begin{pmatrix}
			\frac12 & \frac{1}{2\sqrt{3}} & -\frac{1}{\sqrt{6}} \\
			-\frac12 & \frac{3}{2\sqrt{3}} & 0 \\
			\frac12 & \frac{1}{2\sqrt{3}} & -\frac{1}{\sqrt{6}} \\
			\frac12 & \frac{1}{2\sqrt{3}} & \frac{2}{\sqrt{6}} 
		\end{pmatrix}
		\begin{pmatrix}
			2 & 1 & -1 \\
			0 & \sqrt{3} & \frac{\sqrt{3}}{3} \\
			0 & 0 & \frac{2\sqrt{6}}{3} \\
		\end{pmatrix}
		 = 
		\boldsymbol{QR}
	\end{align*}
	其中,
	\begin{ceqn}
		\begin{gather*}
			\boldsymbol{Q} =
			\begin{pmatrix}
%				\displaystyle{}
				\frac12 & \frac{1}{2\sqrt{3}} & -\frac{1}{\sqrt{6}} \\
				-\frac12 & \frac{3}{2\sqrt{3}} & 0 \\
				\frac12 & \frac{1}{2\sqrt{3}} & -\frac{1}{\sqrt{6}} \\
				\frac12 & \frac{1}{2\sqrt{3}} & \frac{2}{\sqrt{6}} 
			\end{pmatrix}
			\quad
			\boldsymbol{R} =
			\begin{pmatrix}
				2 & 1 & -1 \\
				0 & \sqrt{3} & \frac{\sqrt{3}}{3} \\
				0 & 0 & \frac{2\sqrt{6}}{3} \\
			\end{pmatrix}
		\end{gather*}
	\end{ceqn}
%		\dfrac{分子}{分母}形成正常字号元素
	此即为$\boldsymbol{A}$矩阵的$\boldsymbol{QR}$分解.
	\clearpage
	
	
	\section*{\kaishu 习题24.}
	\subsection*{(1) \kaishu {解:}}
		\begin{ceqn}
			\begin{align*}
				\boldsymbol{A} =
				\begin{pmatrix}
					1 & 0 \\
					0 & 1 \\
					1 & 1
				\end{pmatrix}
			\end{align*}‎
		\end{ceqn} \\
	\indent 显然矩阵$\boldsymbol{A}$为一个$3 \times 2$的矩阵,且$rank(\boldsymbol{A}) = 2.$ \\
	\indent 有 $
		\boldsymbol{A}^H\boldsymbol{A} =
			\begin{pmatrix}
				1 & 0 & 1 \\
				0 & 1 & 1
			\end{pmatrix}
			\begin{pmatrix}
				1 & 0 \\
				0 & 1 \\
				1 & 1
			\end{pmatrix}
		 = 
		 	\begin{pmatrix}
		 		2 & 1 \\
		 		1 & 2 
		 	\end{pmatrix}
		 $ ,\\
	\indent 令 $
		 \lvert \lambda \boldsymbol{E} - \boldsymbol{A}^H\boldsymbol{A} \rvert = 0
		 $ ,即
		 $
		 \lvert \lambda \boldsymbol{E} - \boldsymbol{A}^H\boldsymbol{A} \rvert =
		 \begin{vmatrix}
		 	\lambda - 2 & -1 \\
		 	-1 & \lambda - 2
		 \end{vmatrix}
		 = 
		 \boldsymbol{0}
		 $ ,解得矩阵$\boldsymbol{A}^H\boldsymbol{A}$的特征值$\lambda_1 = 3 , \lambda_2 = 1.$ 则矩阵$\boldsymbol{A}$的非零奇异值为$\sigma_1 = \sqrt{3} , \sigma_2 = 1.$ 记矩阵$\boldsymbol{\Sigma}$为
		 \begin{ceqn}
		 	\begin{align*}
		 		\boldsymbol{\Sigma} =
		 		\begin{pmatrix}
		 			\sqrt{3} & 0 \\
		 			0 & 1 
		 		\end{pmatrix}
		 	\end{align*}‎
		 \end{ceqn} 
	 \indent 求矩阵$\boldsymbol{A}^H\boldsymbol{A}$对应于特征值$3$和$1$的特征向量:
		$$
		\begin{array}{cc}
			(3\boldsymbol{I} - \boldsymbol{A}^H\boldsymbol{A}) x = 0 \qquad & \qquad (\boldsymbol{I} - \boldsymbol{A}^H\boldsymbol{A}) x = 0 \\
			\begin{pmatrix}
				1 & -1 \\
				-1 & 1
			\end{pmatrix} x = 0 &
			\begin{pmatrix}
				-1 & -1 \\
				-1 & -1
			\end{pmatrix} x = 0 \\
			x_1 = \begin{pmatrix}
				1 \\ 1
			\end{pmatrix} &
			x_2 = \begin{pmatrix}
				1 \\ -1
			\end{pmatrix}
		\end{array}				 	
			$$
		\indent 将$x_1,x_2$单位化得
			$\xi_1 = 
			\begin{pmatrix}
				\frac{1}{\sqrt{2}} & \frac{1}{\sqrt{2}}
			\end{pmatrix}^T,
			\xi_2 = 
			\begin{pmatrix}
				\frac{1}{\sqrt{2}} & -\frac{1}{\sqrt{2}}
			\end{pmatrix}^T.
			$ 记矩阵$\boldsymbol{V}$为
			\begin{ceqn}
				\begin{align*}
					\boldsymbol{V} = 
					[\xi_1,\xi_2] = 
					\begin{pmatrix}
						\frac{1}{\sqrt{2}} & \frac{1}{\sqrt{2}} \\
						\frac{1}{\sqrt{2}} & -\frac{1}{\sqrt{2}} 
					\end{pmatrix}
				\end{align*}‎
			\end{ceqn}
		\indent 取矩阵$\boldsymbol{V}$的前$rank(\boldsymbol{A}) = 2$列为矩阵$\boldsymbol{V}_1$:即
			$
			\boldsymbol{V}_1 = \boldsymbol{V} = 
			\begin{pmatrix}
				\frac{1}{\sqrt{2}} & \frac{1}{\sqrt{2}} \\
				\frac{1}{\sqrt{2}} & -\frac{1}{\sqrt{2}} 
			\end{pmatrix}.
			$ \\
		\indent 令矩阵$\boldsymbol{U}_1 = \boldsymbol{A}\boldsymbol{V}_1\boldsymbol{\Sigma}^{-1} = 
			\begin{pmatrix}
				1 & 0 \\
				0 & 1 \\
				1 & 1
			\end{pmatrix}
			\begin{pmatrix}
				\frac{1}{\sqrt{2}} & \frac{1}{\sqrt{2}} \\
				\frac{1}{\sqrt{2}} & -\frac{1}{\sqrt{2}} 
			\end{pmatrix}
			\begin{pmatrix}
				\frac{1}{\sqrt{3}} & 0 \\
				0 & 1 
			\end{pmatrix} = 
			\begin{pmatrix}
				\frac{1}{\sqrt{6}} & \frac{1}{\sqrt{2}} \\
				\frac{1}{\sqrt{6}} & -\frac{1}{\sqrt{2}} \\
				\frac{\sqrt{2}}{\sqrt{3}} & 0
			\end{pmatrix} , $ \\
			将$\boldsymbol{U}_1$扩充为一个$m \times m$亦即$3 \times 3$的酉矩阵$\boldsymbol{U}$:
			\begin{ceqn}
				\begin{align*}
					\boldsymbol{U} = 
					[\boldsymbol{U}_1,\boldsymbol{U}_2] = 
					\begin{pmatrix}
						\frac{1}{\sqrt{6}} & \frac{1}{\sqrt{2}} & u_{13} \\
						\frac{1}{\sqrt{6}} & -\frac{1}{\sqrt{2}} & u_{23} \\
						\frac{\sqrt{2}}{\sqrt{3}} & 0 & u_{33}
					\end{pmatrix}
				\end{align*}‎
			\end{ceqn}
			由于矩阵$\boldsymbol{U}$是酉矩阵,其各个列向量应该是单位正交的,故可列出方程:
				$$	\left\{
					\begin{array}{c}%有了“&=”符号就需要严谨地在大括号里写等号两边的对齐方式!!!
						\frac{1}{\sqrt{6}} \cdot u_{13} + \frac{1}{\sqrt{6}} \cdot u_{23} + \frac{\sqrt{2}}{\sqrt{3}} \cdot u_{33} = 0 \\
						\frac{1}{\sqrt{2}} \cdot u_{13} + (-\frac{1}{\sqrt{2}}) \cdot u_{23} = 0 \\
						u_{13}^2 + u_{23}^2 + u_{33}^2 = 1
					\end{array} 
					\right.
				$$
			解得$\boldsymbol{U}_2 = 
			\begin{pmatrix}
				u_{13} \\ u_{23} \\ u_{33}
			\end{pmatrix} = 
				\begin{pmatrix}
				\frac{1}{\sqrt{3}} \\ \frac{1}{\sqrt{3}} \\ -\frac{1}{\sqrt{3}}
			\end{pmatrix}
			$,故矩阵$\boldsymbol{U}$为:			
			\begin{ceqn}
				\begin{align*}
					\boldsymbol{U} = %这里“\boldsymbol{U}”如果加了“$_$”会使pdf生成报错
					\begin{pmatrix}
						\frac{1}{\sqrt{6}} & \frac{1}{\sqrt{2}} & \frac{1}{\sqrt{3}} \\
						\frac{1}{\sqrt{6}} & -\frac{1}{\sqrt{2}} & \frac{1}{\sqrt{3}} \\
						\frac{\sqrt{2}}{\sqrt{3}} & 0 & -\frac{1}{\sqrt{3}}
					\end{pmatrix}
				\end{align*}
			\end{ceqn}
		\indent \textbf{综上所述},矩阵$\boldsymbol{A}$的奇异值分解为:
			$$
			\boldsymbol{A} = \boldsymbol{U} \begin{pmatrix}
				\boldsymbol{\Sigma} & 0 \\
				0 & 0
			\end{pmatrix} \boldsymbol{V}^H = 
			\begin{pmatrix}
				\frac{1}{\sqrt{6}} & \frac{1}{\sqrt{2}} & \frac{1}{\sqrt{3}} \\
				\frac{1}{\sqrt{6}} & -\frac{1}{\sqrt{2}} & \frac{1}{\sqrt{3}} \\
				\frac{\sqrt{2}}{\sqrt{3}} & 0 & -\frac{1}{\sqrt{3}}
			\end{pmatrix}
			\begin{pmatrix}
				\sqrt{3} & 0 \\
				0 & 1 \\
				0 & 0
			\end{pmatrix}
			\begin{pmatrix}
				\frac{1}{\sqrt{2}} & \frac{1}{\sqrt{2}} \\
				\frac{1}{\sqrt{2}} & -\frac{1}{\sqrt{2}} 
			\end{pmatrix}.
			$$
			\\ \\
		
		
		\subsection*{(3) \kaishu {解:}}
		\begin{ceqn}
			\begin{align*}
				\boldsymbol{A} =
				\begin{pmatrix}
					-1 & 0 & 1 \\
					0 & 1 & 0 \\
					1 & 0 & -1
				\end{pmatrix}
			\end{align*}‎
		\end{ceqn} \\
		\indent 显然矩阵$\boldsymbol{A}$为一个$3 \times 3$的矩阵,且$rank(\boldsymbol{A}) = 2.$ \\
		\indent 有 $
		\boldsymbol{A}^H\boldsymbol{A} =
		\begin{pmatrix}
			-1 & 0 & 1 \\
			0 & 1 & 0 \\
			1 & 0 & -1
		\end{pmatrix}
		\begin{pmatrix}
			-1 & 0 & 1 \\
			0 & 1 & 0 \\
			1 & 0 & -1
		\end{pmatrix}
		= 
		\begin{pmatrix}
			2 & 0 & -2 \\
			0 & 1 & 0 \\
			-2 & 0 & 2
		\end{pmatrix}
		$ ,\\
		\indent 令 $
		\lvert \lambda \boldsymbol{E} - \boldsymbol{A}^H\boldsymbol{A} \rvert = 0
		$ ,即
		$
		\lvert \lambda \boldsymbol{E} - \boldsymbol{A}^H\boldsymbol{A} \rvert =
		\begin{vmatrix}
			\lambda - 2 & 0 & 2 \\
			0 & \lambda - 2 & 0 \\
			2 & 0 & \lambda-2
		\end{vmatrix}
		= 
		\boldsymbol{0}
		$ ,解得矩阵$\boldsymbol{A}^H\boldsymbol{A}$的特征值$\lambda_1 = 4 , \lambda_2 = 1 , \lambda_3 = 0.$ 则矩阵$\boldsymbol{A}$的非零奇异值为$\sigma_1 = 2 , \sigma_2 = 1.$ 记矩阵$\boldsymbol{\Sigma}$为
		\begin{ceqn}
			\begin{align*}
				\boldsymbol{\Sigma} =
				\begin{pmatrix}
					2 & 0 \\
					0 & 1 
				\end{pmatrix}
			\end{align*}‎
		\end{ceqn} 
		\indent 求矩阵$\boldsymbol{A}^H\boldsymbol{A}$对应于特征值$4$、$1$、$0$的特征向量:
			$$
			\begin{array}{ccc}
				(4\boldsymbol{I} - \boldsymbol{A}^H\boldsymbol{A}) x = 0 \qquad & (\boldsymbol{I} - \boldsymbol{A}^H\boldsymbol{A}) x = 0 \qquad & (0\boldsymbol{I} - \boldsymbol{A}^H\boldsymbol{A}) x = 0 \\
				\begin{pmatrix}
					2 & 0 & 2 \\
					0 & 3 & 0 \\
					2 & 0 & 2 
				\end{pmatrix} x = 0 &
				\begin{pmatrix}
					1 & 0 & 2 \\
					0 & 0 & 0 \\
					2 & 0 & -1
				\end{pmatrix} x = 0 & 
				\begin{pmatrix}
					-2 & 0 & 2 \\
					0 & -1 & 0 \\
					2 & 0 & -2 
				\end{pmatrix} x = 0	\\
				x_1 = \begin{pmatrix}
					1 \\ 0 \\ -1
				\end{pmatrix} &
				x_2 = \begin{pmatrix}
					0 \\ 1 \\ 0
				\end{pmatrix} &
				x_3 = \begin{pmatrix}
					1 \\ 0 \\ 1
				\end{pmatrix}
			\end{array}
				$$
		\indent 将$x_1,x_2,x_3$单位化得
		$\xi_1 = 
		\begin{pmatrix}
			\frac{1}{\sqrt{2}} & 0 & -\frac{1}{\sqrt{2}}
		\end{pmatrix}^T,
		\xi_2 = 
		\begin{pmatrix}
			0 & 1 & 0
		\end{pmatrix}^T,
		\xi_3 = 
		\begin{pmatrix}
			\frac{1}{\sqrt{2}} & 0 & \frac{1}{\sqrt{2}}
		\end{pmatrix}^T.
		$ 记矩阵$\boldsymbol{V}$为
		\begin{ceqn}
			\begin{align*}
				\boldsymbol{V} = 
				[\xi_1,\xi_2,\xi_3] = 
				\begin{pmatrix}
					\frac{1}{\sqrt{2}} & 0 & \frac{1}{\sqrt{2}} \\
					0 & 1 & 0 \\
					-\frac{1}{\sqrt{2}} & 0 & \frac{1}{\sqrt{2}} 
				\end{pmatrix}
			\end{align*}‎
		\end{ceqn}
		\indent 取矩阵$\boldsymbol{V}$的前$rank(\boldsymbol{A}) = 2$列为矩阵$\boldsymbol{V}_1$:即
		$
		\boldsymbol{V}_1 = 
		\begin{pmatrix}
			\frac{1}{\sqrt{2}} & 0 \\
			0 & 1 \\
			-\frac{1}{\sqrt{2}} & 0
		\end{pmatrix}.
		$ \\
		\indent 令矩阵$\boldsymbol{U}_1 = \boldsymbol{A}\boldsymbol{V}_1\boldsymbol{\Sigma}^{-1} = 
		\begin{pmatrix}
			-1 & 0 & 1 \\
			0 & 1 & 0 \\
			1 & 0 & -1
		\end{pmatrix}
		\begin{pmatrix}
			\frac{1}{\sqrt{2}} & 0 \\
			0 & 1 \\
			-\frac{1}{\sqrt{2}} & 0
		\end{pmatrix}
		\begin{pmatrix}
			\frac12 & 0 \\
			0 & 1 
		\end{pmatrix} = 
		\begin{pmatrix}
			-\frac{\sqrt{2}}{2} & 0 \\
			0 & 1 \\
			\frac{\sqrt{2}}{2} & 0
		\end{pmatrix} , $ \\
		将$\boldsymbol{U}_1$扩充为一个$m \times m$亦即$3 \times 3$的酉矩阵$\boldsymbol{U}$:
		\begin{ceqn}
			\begin{align*}
				\boldsymbol{U} = 
				[\boldsymbol{U}_1,\boldsymbol{U}_2] = 
				\begin{pmatrix}
					-\frac{\sqrt{2}}{2} & 0 & u_{13} \\
					0 & 1 & u_{23} \\
					\frac{\sqrt{2}}{2} & 0 & u_{33}
				\end{pmatrix}
			\end{align*}‎
		\end{ceqn}
		由于矩阵$\boldsymbol{U}$是酉矩阵,其各个列向量应该是单位正交的,故可列出方程:
		$$	\left\{
		\begin{array}{c}%有了“&=”符号就需要严谨地在大括号里写等号两边的对齐方式!!!
			(-\frac{\sqrt{2}}{2}) \cdot u_{13} + 0 \cdot u_{23} + \frac{\sqrt{2}}{2} \cdot u_{33} = 0 \\
			1 \cdot u_{23} = 0 \\
			u_{13}^2 + u_{23}^2 + u_{33}^2 = 1
		\end{array} 
		\right.
		$$
		解得$\boldsymbol{U}_2 = 
		\begin{pmatrix}
			u_{13} \\ u_{23} \\ u_{33}
		\end{pmatrix} = 
		\begin{pmatrix}
			\frac{\sqrt{2}}{2} \\ 0 \\ \frac{\sqrt{2}}{2}
		\end{pmatrix}
		$,故矩阵$\boldsymbol{U}$为:			
		\begin{ceqn}
			\begin{align*}
				\boldsymbol{U} = %这里“\boldsymbol{U}”如果加了“$_$”会使pdf生成报错
				\begin{pmatrix}
					-\frac{\sqrt{2}}{2} & 0 & \frac{\sqrt{2}}{2} \\
					0 & 1 & 0 \\
					\frac{\sqrt{2}}{2} & 0 & \frac{\sqrt{2}}{2}
				\end{pmatrix}
			\end{align*}
		\end{ceqn}
		\indent \textbf{综上所述},矩阵$\boldsymbol{A}$的奇异值分解为:
		$$
		\boldsymbol{A} = \boldsymbol{U} \begin{pmatrix}
			\boldsymbol{\Sigma} & 0 \\
			0 & 0
		\end{pmatrix} \boldsymbol{V}^H = 
		\begin{pmatrix}
			-\frac{\sqrt{2}}{2} & 0 & \frac{\sqrt{2}}{2} \\
			0 & 1 & 0 \\
			\frac{\sqrt{2}}{2} & 0 & \frac{\sqrt{2}}{2}
		\end{pmatrix}
		\begin{pmatrix}
			2 & 0 & 0 \\
			0 & 1 & 0 \\
			0 & 0 & 0
		\end{pmatrix}
		\begin{pmatrix}
			\frac{1}{\sqrt{2}} & 0 & -\frac{1}{\sqrt{2}} \\
			0 & 1 & 0 \\
			\frac{1}{\sqrt{2}} & 0 & \frac{1}{\sqrt{2}} 
		\end{pmatrix}.
		$$
		
		
		
		
		
		
		
		
		
\end{document}