\documentclass[12pt,fleqn]{article}
%\usepackage[fleqn]{amsmath}默认是左对齐,而且是全局设置,所以之后对于段内公式即使加入\centering或者是都无法做到使公式居中.
\usepackage{ctex}
\usepackage{amsmath}
\usepackage{geometry}
\usepackage{nccmath}
\usepackage{amssymb}
\usepackage{color}

%\usepackage{datetime}
%\newdate{date}{22}{07}{2023}

\linespread{}

\geometry{a4paper,left=1in, right=1in, top=1in, bottom=1in}

\title{\heiti 矩阵论第三章作业}
\author{XXX SZ2305XXX}
\date{2023年7月22日}
%微软雅黑字体:\CJKfamily{zhyahei}

\renewcommand\arraystretch{0.8}
\renewcommand{\baselinestretch}{1.5}




\begin{document}
	\maketitle
	%	矩阵变换过程如下:
	\section*{\kaishu 习题2.}
	\subsection*{(3) \kaishu {解:}}
	%	\large
	\[
	\begin{pmatrix}
		\lambda+1 & \lambda^2+1 & \lambda^2 \\
		3\lambda-1 & 3\lambda^2-1 & \lambda^2+2\lambda \\
		\lambda-1 & \lambda^2-1 & \lambda
	\end{pmatrix}
	\xrightarrow{[1+3(-1)]}
	\begin{pmatrix}
		2 & 2 & \lambda^2-\lambda \\
		3\lambda-1 & 3\lambda^2-1 & \lambda^2+2\lambda \\	
		\lambda-1 & \lambda^2-1 & \lambda
	\end{pmatrix} \\
	\]
	%换行
	\[
	\xrightarrow{[2+3(-3)]}
	\begin{pmatrix}
		2 & 2 & \lambda^2-\lambda \\
		2 & 2 & \lambda^2-\lambda \\	
		\lambda-1 & \lambda^2-1 & \lambda
	\end{pmatrix}
	\xrightarrow{[2+1(-1)]}
	\begin{pmatrix}
		2 & 2 & \lambda^2-\lambda \\
		0 & 0 & 0 \\	
		\lambda-1 & \lambda^2-1 & \lambda
	\end{pmatrix} \\
	\]
	%{\substack{abc \\ \text{above}}}
	%{[2,3],[1(\frac12)]}
	% 这里是草稿纸()
	\[
	\xrightarrow{\substack{[2,3] \\ [1(\frac12)]}}
	\begin{pmatrix}
		1 & 1 & \frac{\lambda^2-\lambda}{2} \\
		\lambda-1 & \lambda^2-1 & \lambda \\	
		0 & 0 & 0
	\end{pmatrix}
	\xrightarrow{[2+1(-\lambda+1)]}
	\begin{pmatrix}
		1 & 1 & \frac{\lambda^2-\lambda}{2} \\
		0 & \lambda^2-\lambda & \lambda-\lambda·\frac{(\lambda-1)^2}{2} \\	
		0 & 0 & 0
	\end{pmatrix} \\
	\]
	\[
	\xrightarrow[\substack{[2+1(-1)] \\ [3+1(-\frac{\lambda^2-\lambda}{2})]}]{}
	\begin{pmatrix}
		1 & 0 & 0 \\
		0 & \lambda^2-\lambda & \lambda-\lambda·\frac{(\lambda-1)^2}{2} \\	
		0 & 0 & 0
	\end{pmatrix}
	\xrightarrow[{[3+2(\frac{-\lambda-1}{2})]}]{}%这里不使用大括号分组会出错,不知道为什么
	\begin{pmatrix}
		1 & 0 & 0 \\
		0 & \lambda^2-\lambda & \lambda \\	
		0 & 0 & 0
	\end{pmatrix} \\
	\]
	\[
	\xrightarrow[\substack{[2,3] \\ [3+2(-\lambda+1)]}]{}
	\begin{pmatrix}
		1 & 0 & 0 \\
		0 & \lambda & 0 \\	
		0 & 0 & 0
	\end{pmatrix}
	\]
	\qquad 即得$\begin{pmatrix}
		\lambda+1 & \lambda^2+1 & \lambda^2 \\
		3\lambda-1 & 3\lambda^2-1 & \lambda^2+2\lambda \\	
		\lambda-1 & \lambda^2-1 & \lambda
	\end{pmatrix}$的$Smith$标准形为$\	\begin{pmatrix}
	1 & 0 & 0 \\
	0 & \lambda & 0 \\	
	0 & 0 & 0
	\end{pmatrix}$.
	\clearpage



	\section*{\kaishu 习题3.}
    \subsection*{(3) \kaishu {解:}}
    %	\large
    矩阵$\begin{pmatrix}
    	\lambda & -1 & 0 & 0 \\
    	0 & \lambda & -1 & 0 \\	
    	0 & 0 & \lambda & -1 \\
    	5 & 4 & 3 & \lambda+2
    \end{pmatrix}$有三阶子式$\begin{vmatrix}
     -1 & 0 & 0 \\
     \lambda & -1 & 0 \\	
     0 & \lambda & -1 
    \end{vmatrix}=-1$,则该矩阵的第1,2,3阶行列式因子均为1,则该矩阵的第1,2,3阶不变因子也均为1.\\[8pt]
    %要手动改行间距最好等定稿再改,不然后续操作会影响
    \indent 
    又$D_4(\lambda)=\begin{vmatrix}
    	\lambda & -1 & 0 & 0 \\
    	0 & \lambda & -1 & 0 \\	
    	0 & 0 & \lambda & -1 \\
    	5 & 4 & 3 & \lambda+2
    \end{vmatrix}=\lambda^4+2\lambda^3+3\lambda^2+4\lambda+5$,其为第四阶行列式因子亦即第四阶初等因子.
    
	\indent 方程$\lambda^4+2\lambda^3+3\lambda^2+4\lambda+5=0$有根:$0.2878\pm1.4161i$,$-1.2878\pm0.8579i$.则原矩阵有初等因子$\lambda-(0.2878+1.4161i)$,$\lambda-(0.2878-1.4161i)$,$\lambda-(-1.2878+0.8579i)$,$\lambda-(-1.2878-0.8579i)$.\par
	\indent \textbf{综上所述},原矩阵的不变因子为1,1,1,$\lambda^4+2\lambda^3+3\lambda^2+4\lambda+5$;初等因子为$\lambda-(0.2878+1.4161i)$,$\lambda-(0.2878-1.4161i)$,$\lambda-(-1.2878+0.8579i)$,$\lambda-(-1.2878-0.8579i)$.\\
		
	\subsection*{(4) \kaishu {解:}}
	矩阵$\begin{pmatrix}
		0 & 0 & 1 & \lambda+2 \\
		0 & 1 & \lambda+2 & 0 \\	
		1 & \lambda+2 & 0 & 0 \\
		\lambda+2 & 0 & 0 & 0
	\end{pmatrix}$有三阶子式$\begin{vmatrix}
		0 & 0 & 1 \\
		0 & 1 & \lambda+2 \\	
		1 & \lambda+2 & 0 
	\end{vmatrix}=-1$,则该矩阵的第1,2,3阶行列式因子均为1,则该矩阵的第1,2,3阶不变因子也均为1.\\
	\indent
	又$D_4(\lambda)=\begin{vmatrix}
		0 & 0 & 1 & \lambda+2 \\
		0 & 1 & \lambda+2 & 0 \\	
		1 & \lambda+2 & 0 & 0 \\
		\lambda+2 & 0 & 0 & 0
	\end{vmatrix}=(\lambda+2)^4$,其为第四阶行列式因子,亦即第四阶初等因子.\\[8pt]
	\indent \textbf{综上所述},原矩阵的不变因子为1,1,1,$(\lambda+2)^4$;初等因子为$(\lambda+2)^4$.
	\clearpage
	
	
	\section*{\kaishu 习题4.}
	\subsection*{\kaishu {解:}}
	矩阵\[A(\lambda)=\begin{pmatrix}
		\lambda & 0 & \ldots & 0 & a_{n} \\
		-1 & \lambda & \ldots & 0 & a_{n-1}\\	
		\vdots & \vdots &  & \vdots & \vdots \\ 
		0 & 0 & \ldots & \lambda & a_{2} \\
		0 & 0 & \ldots & -1 & \lambda+1
	\end{pmatrix}$有$n-1$阶子式$\begin{vmatrix}
		-1 & \lambda & \ldots & 0 \\	
		\vdots & \vdots &  & \vdots & \\ 
		0 & 0 & \ldots & \lambda \\
		0 & 0 & \ldots & -1 
	\end{vmatrix}=-1^{(n-1)}\]%这里使用行内公式符号$会因公式太长了自动换行
	则$A(\lambda)$有$n-1$阶行列式因子与初等因子均为$1$.\\[8pt]
%	\[
		\indent 又 \begin{align*}%这里竟然不要$$符号?!
		\begin{vmatrix} A(\lambda) \end{vmatrix} 
		&= \begin{vmatrix}
		\lambda & 0 & \ldots & 0 & a_{n} \\
		-1 & \lambda & \ldots & 0 & a_{n-1}\\	
		\vdots & \vdots &  & \vdots & \vdots \\ 
		0 & 0 & \ldots & \lambda & a_{2} \\
		0 & 0 & \ldots & -1 & \lambda+1
           \end{vmatrix}\text{,其按最后一列展开}\\ %最后一列展开才会简单!!!                  
    	&= a_n + a_{n-1}\lambda +a_{n-2}\lambda^2 + \ldots + a_2\lambda^{n-2} + (\lambda+a_1)\lambda^{n-1}\\
    	&= a_n + a_{n-1}\lambda +a_{n-2}\lambda^2 + \ldots + a_2\lambda^{n-2} + a_1\lambda^{n-1} + \lambda^n
	\end{align*}
%	\]
	所得即为$A(\lambda)$的第$n$阶行列式因子与不变因子.\\
	\textbf{综上所述},$A(\lambda)$的行列式因子和不变因子均为
%	\centering
%:center 等环境会在上下文产生一个额外间距,而\centering 等命令不产生,只是改变对齐方式.
		\[		\underbrace{1,1,\ldots,1}_\text{$n-1$个},a_n + a_{n-1}\lambda +a_{n-2}\lambda^2 + \ldots + a_2\lambda^{n-2} + a_1\lambda^{n-1} + \lambda^n.	    \]
		\clearpage

	
	
		\section*{\kaishu 习题5.}
	\subsection*{\kaishu {解:}}
%	\[
	\begin{gather*}
	A(\lambda) = %行末的换行符视为一个空格.空格键和Tab键输入的空白字符视为“空格”.连续的若干个空白字符视为一个空格.(一行开头的空格忽略不计.)
	\begin{pmatrix}
	3\lambda+1 & \lambda & 4\lambda-1 \\
	1-\lambda^2 & \lambda-1 &\lambda-\lambda^2 \\
	\lambda^2+\lambda+2 & \lambda & \lambda^2+2\lambda
                 \end{pmatrix}
	\xrightarrow[{[1+2(-3)]}]{}
	 \begin{pmatrix}
	1 & \lambda & 4\lambda-1 \\
	-\lambda^2-3\lambda+4 & \lambda-1 &\lambda-\lambda^2 \\
	\lambda^2-2\lambda+2 & \lambda & \lambda^2+2\lambda
     \end{pmatrix}\\
     \xrightarrow{\substack{[2+1(\lambda^2+3\lambda-4)] \\ [3+1(-\lambda^2+2\lambda-2)]}}
     \begin{pmatrix}
     	1 & \lambda & 4\lambda-1 \\
     	0 & (\lambda-1)(\lambda^2+4\lambda+1) & 2(\lambda-1)(2\lambda^2+7\lambda-2) \\
     	0 & -\lambda(\lambda-1)^2 & -2(\lambda-1)^2(2\lambda-1)
     \end{pmatrix}\\
     \xrightarrow[\substack{[2+1(-\lambda)] \\ [3+1(-4\lambda+1)] \\ [3(\frac12)]}]{}
     \xrightarrow{[3(-1)]}
      \begin{pmatrix}
     	1 & 0 & 0 \\
     	0 & (\lambda-1)(\lambda^2+4\lambda+1) & (\lambda-1)(2\lambda^2+7\lambda-2) \\
     	0 & \lambda(\lambda-1)^2 & (\lambda-1)^2(2\lambda-1)
     \end{pmatrix}\\
%     接下来怎么做???凑公因式!!!
%	  注意:公因式($\lambda-1$)不能消掉哦,这可是不变因子!!!
     \xrightarrow[\substack{[3+2(-2)] \\ [3(-1)]}]{}
     \begin{pmatrix}
     	1 & 0 & 0 \\
     	0 & (\lambda-1)(\lambda^2+4\lambda+1) & (\lambda-1)(\lambda+4) \\
     	0 & \lambda(\lambda-1)^2 & (\lambda-1)^2
     \end{pmatrix}\\
     \xrightarrow[{[2+3(-\lambda)]}]{}
     \begin{pmatrix}
     	1 & 0 & 0 \\
     	0 & (\lambda-1) & (\lambda-1)(\lambda+4) \\
     	0 & 0 & (\lambda-1)^2
     \end{pmatrix}
      \xrightarrow[{[3+2(-\lambda-4)]}]{}
      \begin{pmatrix}
      	1 & 0 & 0 \\
      	0 & (\lambda-1) & 0 \\
      	0 & 0 & (\lambda-1)^2
      \end{pmatrix}
     \end{gather*}
     \indent 另,有
     \begin{gather*}
     B(\lambda) = %行末的换行符视为一个空格.空格键和Tab键输入的空白字符视为“空格”.连续的若干个空白字符视为一个空格.(一行开头的空格忽略不计.)
     \begin{pmatrix}
     	\lambda+1 & \lambda-2 & \lambda^2-2\lambda \\
     	2\lambda & 2\lambda-3 &\lambda^2-2\lambda \\
     	-2 & 1 & 1
     \end{pmatrix}
     %矩阵化简心得:不要把高次方加给低次方,会变复杂!!!要加也最后加!!!
     \xrightarrow[\substack{ [3+2(-\lambda)] \\ [2+1(-\lambda)] \\ [2(-\frac13)]} ]{}
     \begin{pmatrix}
     	\lambda+1 & 1 & 0 \\
     	2\lambda & 1 & -\lambda+\lambda \\
     	-2 & -1 & -\lambda+1
     \end{pmatrix}\\
     \xrightarrow[\substack{ [1,2] \\ [2+1(-\lambda-1)] } ]{}
     \begin{pmatrix}
     	1 & 0 & 0 \\
     	1 & \lambda-1 & -\lambda^2+\lambda \\
     	-1 & \lambda-1 & -\lambda+1
     \end{pmatrix}
      \xrightarrow{\substack{[2+1(-1)] \\ [3+1(1)] \\ [3+2(-1)]}}
      \begin{pmatrix}
      	1 & 0 & 0 \\
      	0 & \lambda-1 & -\lambda(\lambda-1) \\
      	0 & 0 & (\lambda-1)^2
      \end{pmatrix}\\
      \xrightarrow[{[3+2(\lambda)]}]{}
      \begin{pmatrix}
      	1 & 0 & 0 \\
      	0 & (\lambda-1) & 0 \\
      	0 & 0 & (\lambda-1)^2
      \end{pmatrix}
     \end{gather*}
%	\]
    \indent 即$A(\lambda)$与$B(\lambda)$有相同的不变因子,则$A(\lambda)$与$B(\lambda)$相抵.
    \clearpage

	
	\section*{\kaishu 习题6.}
	\subsection*{(1) \kaishu {解:}{\color{red} (答案用的特征值方法也没有讲过)}}
	\begin{gather*}
		\lambda I - A = 
		\begin{pmatrix}
			\lambda-3 & -2 & 5 \\
			-2 & \lambda-6 & 10 \\
			-1 & -2 & \lambda+3
		\end{pmatrix}
		\xrightarrow{\substack{[1,3] \\ [1(-1)]}}
		\begin{pmatrix}
			1 & 2 & -\lambda-3 \\
			-2 & \lambda-6 & 10 \\
			\lambda-3 & -2 & 5
		\end{pmatrix}	\\
		\xrightarrow{\substack{[2+1(2)] \\ [3+1(-\lambda+3)]}}
		\begin{pmatrix}
			1 & 2 & -\lambda-3 \\
			0 & \lambda-2 & -2\lambda+4 \\
			0 & -2\lambda+4 & (\lambda+3)(\lambda-3)+5
		\end{pmatrix}	
		\xrightarrow[\substack{[2+1(-2)] \\ [3+1(\lambda+3)]}]{}
		\begin{pmatrix}
			1 & 0 & 0 \\
			0 & \lambda-2 & -2(\lambda-2) \\
			0 & -2(\lambda-2) & (\lambda-2)(\lambda+2)
		\end{pmatrix}	\\
		\xrightarrow[{[3+2(2)]}]{[3+2(2)]}
		\begin{pmatrix}
			1 & 0 & 0 \\
			0 & \lambda-2 & 0 \\
			0 & 0 & (\lambda-2)^2
		\end{pmatrix}\\[8pt]
		\text{另,有}\\[8pt]%另一种打文字的办法!!!
		\lambda I - B = 
		\begin{pmatrix}
			\lambda-6 & -20 & 34 \\
			-6 & \lambda-32 & 51 \\
			-4 & -20 & \lambda+32
		\end{pmatrix}
		\xrightarrow{\substack{[1,3] \\ [1(-1)]}}
		\begin{pmatrix}
			4 & 20 & -\lambda-32 \\
			-6 & \lambda-32 & 51 \\
			\lambda-6 & -20 & 34
		\end{pmatrix}\\
		\xrightarrow{[2+1(\frac32)]}
		\begin{pmatrix}
			4 & 20 & -\lambda-32 \\
			0 & \lambda-2 & \frac{-3\lambda+6}{2} \\
			\lambda-6 & -20 & 34
		\end{pmatrix}
		\xrightarrow[{[3+2(\frac32)]}]{}
		\begin{pmatrix}
			4 & 20 & -\lambda-2 \\
			0 & \lambda-2 & 0 \\
			\lambda-6 & -20 & 4
		\end{pmatrix}	\\
		\xrightarrow[]{[3+1(1)]}
		\xrightarrow[{[3+1(1)]}]{}
		\begin{pmatrix}
			4 & 20 & -\lambda-2 \\
			0 & \lambda-2 & 0 \\
			\lambda-2 & 0 & 0
		\end{pmatrix}
		\xrightarrow[\substack{[2+1(-5)] \\ [3+1(\frac{\lambda-2}{4})]}]{}
		\begin{pmatrix}
			4 & 0 & 0 \\
			0 & \lambda-2 & 0 \\
			\lambda-2 & -5(\lambda-2) & -\frac14(\lambda-2)
		\end{pmatrix}	\\
		\xrightarrow[\substack{[1(\frac14)] \\ [3(4)]}]{\substack{[3+1(\frac{-\lambda+2}{4})] \\ [3+2(5)]}}
		\begin{pmatrix}
			1 & 0 & 0 \\
			0 & \lambda-2 & 0 \\
			0 & 0 & (\lambda-2)^2
		\end{pmatrix}
	\end{gather*}
	\indent 得$\lambda I - A$与$\lambda I - B$相抵,则$A(\lambda)$与$B(\lambda)$相似.
	\clearpage
	
	
	\section*{\kaishu 习题9.}
	\subsection*{(1) \kaishu {解:}}
	记$A = 
	\begin{pmatrix}
		2 & 6 & -15 \\
		1 & 1 & -5 \\
		1 & 2 & -6
	\end{pmatrix}$,则有
	\begin{gather*}
		\lambda I - A = 
		\begin{pmatrix}
			\lambda-2 & -6 & 15 \\
			-1 & \lambda-1 & 5 \\
			-1 & -2 & \lambda+6
		\end{pmatrix}
		\to
		\begin{pmatrix}
			1 & -2 & \lambda+6 \\
			1 & \lambda-1 & 5 \\
			-\lambda+2 & -6 & 15
		\end{pmatrix}	\\
		\to
		\begin{pmatrix}
			1 & -2 & \lambda+6 \\
			0 & \lambda+1 & -\lambda-1 \\
			0 & -2\lambda-2 & \lambda^2+4\lambda+3
		\end{pmatrix}
		\to
		\begin{pmatrix}
			1 & 0 & 0 \\
			0 & \lambda+1 & -(\lambda+1) \\
			0 & -(2\lambda+1) & (\lambda+1)(\lambda+3)
		\end{pmatrix}	\\
		\to
			\begin{pmatrix}
			1 &  &  \\
			 & \lambda+1 & \\
			 &  & (\lambda+1)^2
		\end{pmatrix}
	\end{gather*}
	\indent 则$A$的初等因子为$\lambda+1$,$(\lambda+1)^2$,故$A$的$Jordan$标准形为\par
	\begin{ceqn}
		\begin{align*}
		J =
		\begin{pmatrix}
			-1 & 0 & 0 \\
			0 & -1 & 1 \\
			0 & 0 & -1
		\end{pmatrix} ‎
		\end{align*}‎
	\end{ceqn}\\
	
	\subsection*{(6) \kaishu {解:}}
	$\text{记}B = 
	\begin{pmatrix}
		1 & 2 & 3 & 4 \\
		0 & 1 & 2 & 3 \\
		0 & 0 & 1 & 2 \\
		0 & 0 & 0 & 1  
	\end{pmatrix} \text{,则有}
		\lambda I - B = 
	\begin{pmatrix}
			\lambda-1 & -2 & -3 & -4 \\
			0 & \lambda-1 & -2 & -3 \\
			0 & 0 & \lambda-1 & -2 \\
			0 & 0 & 0 & \lambda-1  
	\end{pmatrix}\text{,}\\ 
	\indent \text{其有1阶子式-2,2阶子式}
	\begin{vmatrix}
		-3 & -4 \\
		-2 & -3  
	\end{vmatrix} = 1\text{,3阶子式}
	\begin{vmatrix}
		-2 & -3 & -4 \\
		\lambda-1 & -2 & -3  \\
		0 & \lambda-1 & -2
	\end{vmatrix}\\ \indent = -4\lambda(\lambda+1)\text{,}
	\begin{vmatrix}
		\lambda-1 & -2 & -3 \\
		0 & \lambda-1 & -2  \\
		0 & 0 & \lambda-1
	\end{vmatrix} = (\lambda-1)^3\text{,二者互质.}\\ \indent  \text{则其前三阶不变因子均为1.} 
	\text{又有四阶子式}
	\begin{vmatrix}
		1 & 2 & 3 & 4 \\
		0 & 1 & 2 & 3 \\
		0 & 0 & 1 & 2 \\
		0 & 0 & 0 & 1  
	\end{vmatrix} = (\lambda-1)^4.$\\ \par
	\indent 则矩阵$B$的初等因子为$(\lambda-1)^4$,故$B$的$Jordan$标准形为\par
	\begin{ceqn}
		\begin{align*}
			J =
			\begin{pmatrix}
				1 & 1 & 0 & 0 \\
				0 & 1 & 1 & 0 \\
				0 & 0 & 1 & 1 \\
				0 & 0 & 0 & 1
			\end{pmatrix} ‎
		\end{align*}‎
	\end{ceqn}
	\clearpage
		

	\section*{\kaishu 习题10.}
	\subsection*{(1) \kaishu {解:}}
	$A = 
	\begin{pmatrix}
		0 & -4 & 0 \\
		1 & -4 & 0 \\
		1 & -2 & -2
	\end{pmatrix} 
	%\quad
	\text{,则}
		\lambda I - A = 
	\begin{pmatrix}
		\lambda & 4 & 0 \\
		-1 & \lambda+4 & 0 \\
		-1 & 2 & \lambda+2
	\end{pmatrix} $ ,对其进行初等变换:\\
	\[
	\begin{pmatrix}
		\lambda & 4 & 0 \\
		-1 & \lambda+4 & 0 \\
		-1 & 2 & \lambda+2
	\end{pmatrix}
	\xrightarrow{\substack{[3(-1)] \\ [1,3]}}
	\begin{pmatrix}
		1 & -2 & -\lambda-2 \\
		-1 & \lambda+4 & 0 \\
		\lambda & 4 & 0
	\end{pmatrix}
	\xrightarrow{\substack{[2+1(1)] \\ [3+1(\lambda)]}}
	\]
	\[
	\begin{pmatrix}
		1 & -2 & -\lambda-2 \\
		0 & \lambda+2 & -\lambda-2 \\
		0 & 2\lambda+4 & \lambda(\lambda+2)
	\end{pmatrix}
	\xrightarrow[\substack{[2+1(2)] \\ [3+1(\lambda+2)]}]{}
	\xrightarrow{[3+2(-2)]}
	\begin{pmatrix}
		1 & 0 & 0 \\
		0 & \lambda+2 & 0 \\
		0 & 0 & (\lambda+2)^2
	\end{pmatrix}
	\]
	\indent 则$A$的初等因子为$\lambda+2$,$(\lambda+2)^2$,故$A$的$Jordan$标准形为
%	\[
	\begin{ceqn}
		\begin{align*}
			J =
			\begin{pmatrix}
				-2 & 0 & 0 \\
				0 & -2 & 1 \\
				0 & 0 & -2
			\end{pmatrix}
		\end{align*}‎
	\end{ceqn}
%	\]
	\indent 要求变换矩阵$P$使得$P^{-1}AP=J$,即求$P=(p_1,p_2,p_3)$满足$AP=PJ$,带入$A$、$P$、$J$得:
	\begin{ceqn}
		\begin{align*}
			(Ap_1,Ap_2,Ap_3) = (p_1,p_2,p_3)
			\begin{pmatrix}
				-2 & 0 & 0  \\
				0 & -2 & 1  \\
				0 & 0 & -2  
			\end{pmatrix} ‎
		\end{align*}‎
	\end{ceqn}
	\indent 比较上式两边得
	\begin{ceqn}
		\begin{align*}
%			 \left\{
			 \begin{cases}
%	\begin{array}{rl}	%创建一个包含两列的数组,r表示右对齐,l表示左对齐.这意味着数组中的第一列将右对齐,第二列将左对齐
				Ap_1 = -2p_1,\\
				Ap_2 = -2p_2,\\
				Ap_3 = p_2 - 2p_3.
			\end{cases} 
%			\right.
		\end{align*}‎
	\end{ceqn}
	\indent 由此可见,$p_1$,$p_2$是$A$的对应于特征值-2的两个线性无关的特征向量.\par
	\indent 从方程组
	\begin{ceqn}
		\begin{align*}
			(2I + A) x = 0 
		\end{align*}‎
	\end{ceqn}
	可求得两个线性无关的特征向量$\xi = \begin{pmatrix}
	0 \\ 0 \\1
	\end{pmatrix}$,$\eta = \begin{pmatrix}
	2 \\ 1 \\0
	\end{pmatrix}$. \\
	\indent 取$p_1 = \xi$,$p_2 = k_1\xi + k_2\eta$,其满足方程 $Ap_3 = p_2 - 2p_3$ 有解,记$p_3 = (x_1,x_2,x_3)^T$,即
	\begin{ceqn}
		\begin{align*}
			\begin{pmatrix}
				2 & -4 & 0 \\
				1 & -2 & 0 \\
				1 & -2 & 0
			\end{pmatrix}
			\begin{pmatrix}
				x_1 \\ x_2 \\ x_3 
			\end{pmatrix}
			=
			\begin{pmatrix}
				2k_2 \\ k_2 \\ k_1 
			\end{pmatrix}
		\end{align*}‎
	\end{ceqn}
	有解.容易看出,当$k_2 = k_1$时方程有解,且其解为
	\begin{ceqn}
		\begin{align*}
			x_1 - 2x_2 = k_1 \\
			x_3\in\mathbb{R}
		\end{align*}‎
	\end{ceqn}
	其中$k_1$为任意非零常数.取$k_1=k_2=1$,可得$p_1 = \xi = \begin{pmatrix}
		0 \\ 0 \\1
	\end{pmatrix}$,$p_2 = k_1\xi + k_2\eta = \begin{pmatrix}
	2 \\ 1 \\1
	\end{pmatrix}$,取$p_3 = \begin{pmatrix}
	3 \\ 1 \\0
	\end{pmatrix}$,则$P = \begin{pmatrix}
	0 & 2 & 3 \\ 0 & 1 & 1 \\ 1 & 1 & 0
	\end{pmatrix}$使得$P^{-1}AP=\begin{pmatrix}
	-2 & 0 & 0 \\
	0 & -2 & 1 \\
	0 & 0 & -2
	\end{pmatrix}$即为所求.\\[8pt]
	
	
	\subsection*{(2) \kaishu {解:}{\color{red} (老师后来讲了不要用特征值这种方法)}} 
	$A = 
	\begin{pmatrix}
		1 & -3 & 4 \\
		4 & -7 & 8 \\
		6 & -7 & 7
	\end{pmatrix} 
	\text{,则}
	\lambda I - A = 
	\begin{pmatrix}
		\lambda-1 & 3 & -4 \\
		-4 & \lambda+7 & -8 \\
		-6 & 7 & \lambda-7
	\end{pmatrix} $ \text{,有}
	\begin{ceqn}
	\begin{align*}
		\begin{vmatrix}
			\lambda I - A
		\end{vmatrix} =
		\begin{vmatrix}
			\lambda-1 & 3 & -4 \\
			-4 & \lambda+7 & -8 \\
			-6 & 7 & \lambda-7
		\end{vmatrix} =
			(\lambda+1)^2(\lambda-3)
	\end{align*}‎
	\end{ceqn}
	因此,$A$有特征值$3$,$-1$,$-1$.\\
	\indent 又$A+I=
	\begin{pmatrix}
		2 & -3 & 4 \\
		4 & -6 & 8 \\
		6 & -7 & 8
	\end{pmatrix}$,显然$r(A+I)=2$,因此$A$的特征值$-1$只对应1个线性无关的特征向量.故$A$的$Jordan$标准形为
	\begin{ceqn}
	\begin{align*}
		J =
		\begin{pmatrix}
			3 & 0 & 0 \\
			0 & -1 & 1 \\
			0 & 0 & -1
		\end{pmatrix}
	\end{align*}‎
	\end{ceqn}
	\[\text{{\color{red} (老师后来讲了不要用特征值这种方法。这里后期再算一遍。点击此处插入文字)}}\]
	对于方程$(3I-A)x=0$,即$
	\begin{pmatrix}
		2 & 3 & -4 \\
		-4 & 10 & -8 \\
		-6 & 7 & -4
	\end{pmatrix}
	\begin{pmatrix}
		x_1 \\ x_2 \\ x_3
	\end{pmatrix} = 0$,有解
	$\xi=\begin{pmatrix}
		1 \\ 2 \\ 2 
	\end{pmatrix}$ ;\\
	对于方程$(-I-A)x=0$,即$
	\begin{pmatrix}
		-2 & 3 & -4 \\
		-4 & 6 & -8 \\
		-6 & 7 & -8
	\end{pmatrix}
	\begin{pmatrix}
		x_1 \\ x_2 \\ x_3
	\end{pmatrix} = 0$,有解
	$\eta=\begin{pmatrix}
		1 \\ 2 \\ 1 
	\end{pmatrix}$. \\
	设可逆矩阵$P=(p_1,p_x,p_3)$满足$P^-1AP=J$,即
	\begin{ceqn}
		\begin{align*}
			(Ap_1,Ap_2,Ap_3) = (p_1,p_2,p_3)
			\begin{pmatrix}
				3 & 0 & 0  \\
				0 & -1 & 1  \\
				0 & 0 & -1  
			\end{pmatrix} ‎
		\end{align*}‎
	\end{ceqn}
	\indent 比较上式两边得
	\begin{ceqn}
		\begin{align*}
			\begin{cases}
					Ap_1 = 3p_1,\\
					Ap_2 = -p_2,\\
					Ap_3 = p_2 - p_3.
				\end{cases} 
			\end{align*}‎
		\end{ceqn}
		\indent 由此可见,$p_1$是$A$对应于特征值3的一个线性无关的特征向量,$p_2$是$A$的对应于特征值-1的一个线性无关的特征向量.故可取$p_1=\xi=\begin{pmatrix}
			1 \\ 2 \\ 2 
		\end{pmatrix}$,$p_2=\eta=\begin{pmatrix}
		1 \\ 2 \\ 1 
		\end{pmatrix}$,$p_3$满足方程$Ap_3 = p_2 - p_3$ 有解,记$p_3 = (x_1,x_2,x_3)^T$,即
		\begin{ceqn}
			\begin{align*}
				\begin{pmatrix}
					2 & -3 & 4 \\
					4 & -6 & 8 \\
					6 & -7 & 8
				\end{pmatrix}
				\begin{pmatrix}
					x_1 \\ x_2 \\ x_3 
				\end{pmatrix}
				=
				\begin{pmatrix}
					1 \\ 2 \\ 1 
				\end{pmatrix}
			\end{align*}‎
		\end{ceqn}
		有解.而方程显然有解,且其解为
		\begin{ceqn}
			\begin{align*}
				-2 x_1 + x_2 = 1
			\end{align*}‎
		\end{ceqn}
		取自由变量$x_3=1$,解得$x_1=0 \text{,} x_2 = 1$,则$P = \begin{pmatrix}
			1 & 1 & 0 \\ 2 & 2 & 1 \\ 2 & 1 & 1
		\end{pmatrix}$使得$P^{-1}AP=\begin{pmatrix}
			3 & 0 & 0 \\
			0 & -1 & 1 \\
			0 & 0 & 1
		\end{pmatrix}$即为所求.\\
	
	
	\clearpage
	
	
	\section*{\kaishu 习题14.}
	\subsection*{(1) \kaishu {解:}}
		记矩阵为$A$
		\begin{gather*}
			\lambda I - A = 
			\begin{pmatrix}
				\lambda-3 & -1 & 1 \\
				0 & \lambda-2 & 0 \\
				-1 & -1 & \lambda-1
			\end{pmatrix}
			\xrightarrow{\substack{[3(-1)] \\ [1,3]]}}
			\begin{pmatrix}
				1 & 1 & -\lambda+1 \\
				0 & \lambda-2 & 0 \\
				\lambda-3 & -1 & 1
			\end{pmatrix}	\\
			\xrightarrow{[3+1(-\lambda+3)]}
			\begin{pmatrix}
				1 & 1 & -\lambda+1 \\
				0 & \lambda-2 & 0 \\
				0 & -\lambda+2 & \lambda^2-4\lambda+4
			\end{pmatrix}	
			\xrightarrow[{[2+1(-1)]}]{}
			\begin{pmatrix}
				1 & 0 & 0 \\
				0 & \lambda-2 & 0 \\
				0 & -(\lambda-2) & (\lambda-2)^2
			\end{pmatrix}	\\
			\xrightarrow{\substack{[3+1(\lambda-1) \\ [3+2(1)]]}}
			\begin{pmatrix}
				1 & 0 & 0 \\
				0 & \lambda-2 & 0 \\
				0 & 0 & (\lambda-2)^2
			\end{pmatrix}
		\end{gather*}
		\indent 由书本定理3.6.5知,$A$的最小多项式为$(\lambda-2)^2$.\\
			
	\subsection*{(2) \kaishu {解:}}
		记矩阵为$B$
		\begin{gather*}
			\lambda I - B = 
			\begin{pmatrix}
				\lambda-4 & 2 & -2 \\
				5 & \lambda-7 & 5 \\
				6 & -7 & \lambda+4
			\end{pmatrix}
			\xrightarrow{[1,2]}
			\xrightarrow{\substack{[3+1(1)] \\ [2+1(\frac{-\lambda+4}{2})]}}
			\begin{pmatrix}
				2 & 0 & 0 \\
				\lambda-7 & \frac{-\lambda^2+11\lambda-18}{2} & \lambda-2 \\
				-7 & \frac{7\lambda-16}{2} & \lambda-3
			\end{pmatrix}\\
			\xrightarrow[\substack{[1(\frac12)] \\ [2(2)]}]{\substack{[2+1(\frac{-\lambda+7}{2})] \\ [3+1(\frac72)]}}
			\begin{pmatrix}
				1 & 0 & 0 \\
				0 & -\lambda^2+11\lambda-18 & \lambda-2 \\
				0 & 7\lambda-16 & \lambda-3
			\end{pmatrix}
			\xrightarrow[]{{[2-3(1)]}}
			\xrightarrow[{[2,3]}]{}
			\begin{pmatrix}
				1 & 0 & 0 \\
				0 & 1 & -\lambda^2+4\lambda-2 \\
				0 & \lambda-3 & 7\lambda-16
			\end{pmatrix} \\
			\xrightarrow{[3+2(-\lambda+3)]}
			\xrightarrow[{[3+2(\lambda^2-4\lambda+2)]}]{}
			\begin{pmatrix}
				1 & 0 & 0 \\
				0 & 1 & 0 \\
		    	0 & 0 & (\lambda-2)(\lambda^2-5\lambda+11)
			\end{pmatrix}	\\
		\end{gather*}
		\indent 由书本定理3.6.5知,$B$的最小多项式为$(\lambda-2)(\lambda^2-5\lambda+11)$.\\
	
		\clearpage
	
	
	\section*{\kaishu 习题18.}
	\subsection*{\kaishu {证:}}
	记$\varphi(\lambda)=\lambda^2-5\lambda+6$,则$\varphi(\lambda)$是$A$的化零多项式.由书本定理3.6.2知$A$的最小多项式$m(\lambda)$整除$\varphi(\lambda)$.因为$\varphi(\lambda)$没有重根,所以$m(\lambda)$也没有重根.根据书本定理3.6.6知$A$相似于对角矩阵.
	
	
	
	
	
	
	
	
	\end{document}
